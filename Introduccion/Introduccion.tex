\pagenumbering{arabic}
Encontrar los fundamentos del funcionamiento de los objetos materiales que componen la naturaleza ha sido una de las tareas de las que se ha ocupado la humanidad. Esta línea de investigación inicio en la química del siglo XIX con el modelo atómico de \citet{dalton} y pasó a ser parte de la física tras el descubrimiento de la radioactividad por %Antoine Henri 
Becquerel [1896] y del electrón por Thomson [1906].

A inicios del siglo XX el área de Física de Partículas Elementales se forma como campo independiente junto con el establecimiento de la composición del núcleo atómico y con el advenimiento de los aceleradores. Ésta se establece entonces como la ciencia que estudia los componentes elementales de la materia y las interacciones entre ellos. También se la conoce como Física de Altas Energías debido a la escala del sistema físico sobre el que se teoriza modelando el comportamiento de la materia.

De forma general, el área de la cosmológica divide la materia en dos grupos: bariónica y oscura. La Física de Altas Energías describe la materia bariónica según la teoría del Modelo Estándar de Partículas o \ME ~ (\textbf{S}tandard \textbf{M}odel), pero ha pesar de mostrar una alta correspondencia con los datos experimentales, pero aún ahora, está incompleto. Este no explica el origen cuántico de la gravedad, siendo esta, una de las preguntas más importantes de la física de partículas contemporánea, la existencia de materia no bariónica o oscura, la asimetría de materia-antimateria o el origen de las masas de neutrinos, los problemas de jerarquía al introducir partículas con masas a través del proceso de ``ruptura espontánea de simetría electrodébil'', entre otras. 


Entre los problemas sin solución a los que se enfrenta el \ME, el solo poder explicar el 15.45\% de la materia del universo se convierte en el objeto de estudio de muchas investigaciones modernas. En las últimas décadas, se han propuesto nuevas teorías para explicar la materia oscura, muchas partículas han sido propuestas como posibles candidatos, entre ellas las más populares se encuentran: los axiones \citep{axion_2019}, las partículas masivas con interacción  débil \WIMPs ~(\textbf{W}eakly \textbf{I}nteracting \textbf{M}assive \textbf{P}articles) \citep{wimps_2018} o con interacción fuerte \textbf{SIMPs} (\textbf{S}trongly \textbf{I}nteracting \textbf{M}assive \textbf{P}articles) \citep{simps_2019} y las partículas supersimétricas o \SUSY~(\textbf{SU}per\textbf{SY}mmetry) \citep{susy_2020}. 

Entre las propuestas existentes sobre la composición de la materia oscura, las partículas supersimetrías o \SUSY ~son los candidatos más populares en los estudios de física de partículas. Esta supersimetría hipotética relaciona las propiedades de los bosones y los fermiones, y a pero a pesar de estar por ser verificada experimentalmente, es parte fundamental de muchos modelos teóricos como el \textbf{N}\MSSM ~ (\textbf{N}ext-to-\textbf{M}inimal \textbf{S}upersymmetric \textbf{S}tandard \textbf{M}odel) \citep{MSSM_1, MSSM_2} y los modelos de supersimetría en el sector oscuro \textbf{Dark-SUSY} \citep{LB}. En esta, se teoriza un conjunto de criterios de búsqueda destinados a minimizar los eventos de fondo sin dejar de ser independientes de los modelos utilizados suponiendo el origen de la ruptura espontánea \textbf{U(1)} descrito en la referencia \cite{dark_1, dark_2}, este resultado del acoplamiento débil de unos fotones oscuros $\gamma_D$ a sus homólogos del \ME ~ a través de un parámetro de mezcla cinética $\epsilon$. %\cite{bb38,bb39}.

%la fracción de decaimiento $\gamma_D\rightarrow \mu^+\mu^-$ puede variar en %ser tan grande como 45\% esto dependencia de la masa $m_{\gamma_D}$. 

En los modelos \textbf{N}\MSSM, dos de los tres bosones de Higgs neutros pares $h_1$ o $h_2$ pueden descomponerse en uno de los dos bosones de Higgs neutros impares $n_{1,2}$ a través de $h_{1,2} \rightarrow  2n_1$. Al unificarse con la teoría del sector oscuro, se teoriza que el neutralino más ligero $n_1$ que se encuentra en el espectro visible de \SUSY ~ya no es estable y puede descomponerse a través de procesos como $n_1  \rightarrow n_D + \gamma_D$ , donde $n_D$ es un fermión oscuro (neutralino oscuro) que escapa a la detección con los instrumentos existentes actuales, está búsqueda superpuesta de teorías es conocida como \textbf{Dark-}\SUSY~ o \textbf{N}\MSSM\textbf{D}. Entonces, bajo la suposición de que $\gamma_D$ solo puede descomponerse en partículas \textbf{SM}, alternativamente, muchas líneas de investigación realizan exploraciones para los posibles decaimientos $h \rightarrow 2n_1$. En estas exploraciones predichas por \textbf{N}\MSSM ~, se incluye $4\mu$ \citep{cms_collaboration_search_2016,cms_collaboration_search_2013} como un posible estado final, contribuyendo este análisis a un cuerpo más grande existente dedicado a la búsqueda de nuevos bosones.

Dado que el experimento \LHC ~ tiene entre sus objetivos una amplia gama de búsquedas de Higgs, entre ellas, aquellas resultado de extensiones del Modelo Estándar Supersimétrico Mínimo \MSSM~ y otros modelos \SUSY, de aquí que sea adecuada para la exploración de los modelos compatibles o derivados como es el \textbf{Dark-}\SUSY. Pero, para probar está teoría, es necesaria una medición cuidadosa de las colisiones en un subconjunto particular de la población de datos. Es necesario que los equipos de análisis puedan calcular con precisión cuántos eventos se esperarían de los procesos del \ME~ en ese subconjunto y, de manera similar, cuántos eventos cabría esperar de la teoría particular de la nueva física en la que uno está interesado. Con estos cálculos en mano, los analistas pueden mirar los datos reales observados y realizar un análisis estadístico que indicara si la teoría particular es favorecida por los datos, normalmente dicho análisis se define mediante un complejo análisis basada en software. %Todo lo anterior es el resultado, de un gran proceso de análisis, gran parte del trabajo en el desarrollo de un teoría consiste en crear un respaldo en datos que contiene la mayor cantidad de información sobre la teoría estudiada, así como también en hacer los cálculos precisos del \ME.

La exploración de como la teoría \textbf{Dark-}\SUSY ~ pueda materializarse en un subconjunto de la población de datos obtenidos experimentalmente es parte importante del proceso de investigación, para esto, se hace necesario calcular la estimación del efecto de la nueva teoría y realizar el análisis estadístico para decidir si es viable frente a los datos observados. De aquí que la simulación de los distintos procesos físicos que se puedan generar en el \LHC ~ y la respuesta del detector es parte necesaria para poder optimizar y estimar el desempeño de los diferentes análisis. Además, de esta manera, se permite que las estrategias utilizadas en la identificación de partículas, puedan ser desarrolladas con anterioridad a la toma de datos y las eficiencias de los algoritmos pueden ser puestos a prueba. La preparación de las búsquedas de está nueva física, necesita una simulación detallada del detector para estimar su potencial de descubrimiento y para desarrollar métodos óptimos para medir las propiedades de las partículas.

De aquí que, una vez que los datos de colisiones reales están disponibles, los simulados resultan necesarios para poder encontrar desviaciones del \ME. La estructura de los eventos de colisiones de altas energías son realmente complejos y no predecibles de primeros principios. Los generadores de eventos como el Madgraph, permiten separar el problema en varios pasos más simples, algunos de los cuales pueden ser descriptos por primeros principios, y otros necesitan ser basados en modelos apropiados con parámetros ajustados a los datos como es el caso del método de \MC ~(\textbf{M}onte \textbf{C}arlos). Un aspecto central de los generadores es que proveen una descripción del estado final para poder construir cualquier observable y compararlos con los datos de colisiones reales.

En los estudios de nueva física en el experimento \LHC, en particular los que predicen la producción de nuevas partículas, son bastantes relevantes dado que se aproxima la etapa de actualización correspondiente al de Alta Luminosidad, donde se espera lograr una acumulación de datos con una frecuencia 10 veces mayor en la que se estaba operando. Lo anterior indica que la probabilidad de detección de nuevas señales será mucho mayor ya que se logrará alcanzar un rango de energía mayor y una cantidad de datos igualmente superior. Usualmente la probabilidad de producción de estas partículas exóticas es baja por lo que se requiere de una cantidad grande de datos para poder observar dicha producción. El período de Alta Luminosidad está programado para empezar a partir del año 2024 o 2025, sin embargo desde este la actualidad se está trabajando en la actualización del detector, en métodos de análisis y en estrategias que ayuden a optimizar la búsqueda de nueva física. Además, dado que los modelos teóricos que predicen la formación de partículas de materia oscura no han sido explorados ampliamente, en gran medida por falta de datos experimentales que permitan alcanzar el espacio fase que dichos modelos predicen para esas partículas. Por todo ello el funcionamiento del Gran Acelerador de Hadrones y sus proyecciones en cuanto a recolección de datos en los próximos años constituye una oportunidad perfecta para explotar con mayor intencionalidad el estudio de dichos modelos en aras de descubrir una nueva señal de fácil interpretación en el contexto de los modelos propuestos. 

El objetivo general de este proyecto es:\\
Estudiar las propiedades del fotón oscuro correspondiente al decaimiento $\gamma_D \rightarrow 2\mu$ del modelo \textbf{Dark-}\SUSY ~ por medio de simulación de \MC, recreando la respuesta de los detectores del experimento \CMS.

Para lograrlo, se lleva a cabo los objetivos particulares:
\begin{itemize}
\item Caracterizar el modelo \textbf{Dark-}\SUSY~por medio de su implementación en paquetes de simulación de altas energías Madgraph5, pythia8 y Delphes3.
\item Análisis de las propiedades de las partículas del decaimiento correspondiente al modelo, bajo diferentes condiciones de simulación.
\item Desarrollar rutinas de extracción y análisis de la información de simulación, en los lenguajes de programación python y C${++}$. 

%eficiencia de detección de las partículas producidas del decaimiento del fotón oscuro, centrándose en estudiar el decaimiento a pares de muones de signo opuesta, exploración de propiedades como función de tiempo de vida del fotón oscuro y optimización de la selección de eventos.
\item Comparación de los resultados obtenidos usando la configuración del detector \CMS ~ Run-2 y \CMS~ Alta Luminosidad.
\end{itemize}


%
%ículas interaccionan débilmente con las partículas del modelo estándar, es decir con la materia visible, por lo que su detección se dará de forma indirecta, o en otras palabras, por su decaimiento a partículas conocidas del modelo estándar~\cite{Curtin2015}. Adicionalmente al estudio del modelo teórico se pretende trabajar en la parte experimental, la cual consiste en el estudio de la respuesta del detector al paso de las nuevas partículas y de esta manera extraer observables experimentales como son la energía de las partículas, el momento y la trayectoria, eficiencia de identificación, resoluciones, entre otros que permitan distinguir el proceso de se\~nal de los procesos de ruido. La parte experimental es fundamental ya que sin una buena estrategia de selección de datos, técnicas de supresión de ruido y optimización de la señal sería imposible la observación de nueva física. En este proyecto se considera el detector CMS del CERN como el aparato experimental que proporcionará los datos de estudio, ya sea por simulación o por uso de datos reales.
%
%
%%% METODOLOGIA
%
%%{\let\clearpage\relax\chapter{Metodología}}
%
%Con el fin de alcanzar los objetivos planteados se plantea la siguiente secuencia de actividades:
