%\begin{textblock}{9}(2.5,-3)
%\begin{flushright}
%\setlength{\baselineskip}{15pt}
%\textblockrulecolour{white}
%~
%
%`` La discrepancia entre lo que se esperaba y lo que se ha observado ha aumentado a lo largo de los años, y nos estamos esforzando cada vez más por llenar el vacío
%''\\[.5cm]
%\textit{Jeremiah P. Ostriker}
%\end{flushright}
%\end{textblock}

En este capítulo se realiza una descripción del experimento \CMS, se definen conceptos básicos correspondientes a la Física de Altas Energías experimental. Se introducen las herramientas personalizadas para el trabajo de simulación, análisis y caracterización en el área de partículas, entre estas se encuentras \ROOT, {Delphes}, {Pythia8} o {Madgraph}. Además, se introducen elementos comparativos para la caracterización del experimento en las configuración actual Run-2 y en la prevista en el futuro correspondiente a Alta Luminosidad.

%necesarios para el cumplimiento de los objetivos, estos serán descritos para su mejor comprensión en el proceso de simulación, caracterización y análisis de resultados. Además se resumirán los resultados que servirán de antecedentes para la investigación.




