\begin{textblock}{9}(2.5,-3)
\begin{flushright}
\setlength{\baselineskip}{15pt}
\textblockrulecolour{white}
~

``Mi investigación en la física ha consistido en simplemente examinar cantidades matemáticas de un tipo que los físicos usan y tratar de relacionarlas de una manera interesante.''\\[.5cm]
\textit{Paul A. M. Dirac.}
\end{flushright}
\end{textblock}

En este capítulo se presentan todos los proyectos de simulación, tratamiento y análisis necesarios para cumplir con el objetivo de esta investigación, estos son programados en \textbf{python} y \textbf{C++}, la explicación de estas herramientas darán las bases para profundizar en los resultados obtenidos y su respectiva discusión.