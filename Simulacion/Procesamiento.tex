

Para poder realizar los estudios requeridos se hace necesario acceder a la información contenida en los archivos $\textsf{*.root}$ de forma eficiente\footnote{La descripción del contenido del árbol de datos de nuestros archivos se puede observar en el enlace \href{https://cp3.irmp.ucl.ac.be/projects/delphes/wiki/WorkBook/RootTreeDescription}{https://\-cp3.\-irmp.\-ucl.\-ac.\-be/\-pro\-jects/\-delphes/\-wiki/\-Work\-Book/\-Root\-Tree\-Des\-crip\-tion}}. Se programa la clase $\textsf{classDarkSUSY.py}$ como interpretador externo al entorno predeterminado de \ROOT ~para poder acceder a la información pertinente a la investigación, está es creada en python y permite fácilmente extraer información del archivo original $\textsf{*.root}$ (se hace uso de las paqueterías $\textsf{pyroot}$). %Esta clase (ver Fig. \ref{class_darksusy1}) además de permitir acceder a los datos, procesa la información para hacer la reconstrucción más probable de la masa del fotón oscuro partiendo de la información muónica, el diagrama de la figura muestra el procedimiento utilizado.

\begin{figure}[!h]
\centering
\includegraphics[width=1\textwidth]{Simulacion/imagenes/class_darksusy.png}
\caption[Estructura del proyecto interpretador de la información contenida en los archivos $\textsf{*.root}$.]{Estructura del proyecto interpretador de la información contenida en los archivos $\textsf{*.root}$\footnote{Página del proyecto: \href{https://github.com/franky8939/DarkSUSY/blob/master/00-Convertidores/genera_h5.py}{https://github.com/franky8939/\-Dark\-SUSY/\-blob/\-mas\-ter/\-00-\-Con\-ver\-ti\-do\-res/\-ge\-ne\-ra\_h5.\-py}.}. }
\label{class_darksusy1}
\end{figure}
\footnotetext{Página del proyecto: \href{https://github.com/franky8939/DarkSUSY/blob/master/modules/darkSUSY/classDarkSUSY.py}{https://\-github\-.com/\-franky8939/\-DarkSUSY/\-blob/\-master/\-modules/\-darkSUSY/\-class\-Dark\-SUSY.\-py}}
Dada la gran cantidad de información y archivos a procesar para el análisis estadístico incluso ante un acceso eficiente, la gran dispersión de la información hace que los procesos de recolección de datos sea lento y con altos requerimientos de memoria, la forma en que se abordo esta dificultad fue incorporar la información solicitada en un mismo archivo de tipo \textbf{HDF5} (\textbf{H}ierarchical \textbf{D}ata \textbf{F}ormat) la cual posee una librería de propósito general con un formato de ficheros para el almacenamiento de datos científicos, ver Fig. \ref{class_darksusy1}. 
%\begin{figure}[!h]
%\centering
%\includegraphics[width=1\textwidth]{Simulacion/imagenes/class_darksusy3.png}
%\caption[Estructura de los metadatos con la información filtrada.]{Estructura de los metadatos con la información filtrada\footnotemark.}
%\label{class_darksusy2}
%\end{figure}
Este formato de datos \textbf{HDF5} fue creado para atender las necesidades de científicos e ingenieros que trabajan en entornos de computación de altas prestaciones que requieran un uso intensivo de datos, de aquí que este predeterminado para que sea muy eficiente en el almacenaniento y el acceso. 

