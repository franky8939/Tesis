Para entender la señal del proceso \MSSM\textbf{D} (estos procesos corresponden con la descomposición según lo muestra el diagrama de la Fig. \ref{fig:sketch_darksector}), siendo el objetivo de estudio en esta investigación, se hace necesario su caracterización antes y después de simular su paso por las diferentes configuraciones del detector. Conocer la morfología de la señal real y la reconstruida por el detector nos permitirá comprender la teoría y como está es visualizada por el experimento \CMS.

\subsection{Variación del contenido muónico}

Se hace hace necesario investigar el contenido muónico de la señal \MSSM\textbf{D}~ bajo las diferentes condiciones de generación, para hacer referencia a estas condiciones iniciales con las que se generó la señal, se hará uso del vector:
\begin{equation}
\vec{\alpha} = (m_{n_1}, m_{n_D}, m_{\gamma_D}, c\tau_{\gamma_D})
\end{equation}
además el número de partículas $p$ en el i-ésimo evento generado está definido por:
\begin{equation}\label{numero_particulas}
n_i^{(p,k)} \equiv n_i^{(p,k)} (\vec{\alpha})
\end{equation}
donde 
$k =$ \begin{small}$\textsf{R2},~\textsf{HL}, ~\textsf{True}$\end{small}  declara la presencia del detector y su configuración, 
$i = 1, \ldots, N_{e}$ hace referencia al evento y 
$p = \mu^\pm, ~ \gamma_D, ~n_D, ~n_1$ partícula caracterizada.
%\begin{tabular}{lll}
%$k$ & $= \textsf{R2},~\textsf{HL}, ~\textsf{True}$ & declara la presencia del detector y su configuración.\\
%$i$ & $ = 1, \ldots, N_{e}$ & hace referencia al evento.\\
%$p$ & $ = \mu^\pm, ~ \gamma_D, ~n_D, ~n_1$ & partícula caracterizada.
%\end{tabular}
Definiendo a $f^{(p, k)}_\textsf{e} (x)\equiv f^{(p, k)}_\textsf{e} (x; \vec{\alpha})$ como la fracción del total de eventos poseedores de un número $x$ de partículas tipo $p$ de la señal generada bajo las condiciones iniciales $\vec{\alpha}$ y con la configuración del detector $k$ se tiene entonces:
%\mathbb{M} 
\begin{equation}\label{fe}
f^{(p, k)}_\textsf{e} (x)  = \dfrac{\sum\limits_{i=1}^{N_e} \delta_{x,n_i^{(p,k)}}}{\sum\limits_{i=1}^{N_e} \sum\limits_{x=0}^\infty \delta_{x,n_i^{(p,k)}}} = \dfrac{1}{N_e} \sum_{i=1}^{N_e} \delta_{x,n_i^{(p,k)}}
\end{equation}
donde $x\in\mathbb{N}$ pertenece al grupo de lo números naturales, $\delta_{x,n_i^{(p,k)}}$ es la función delta de Kronecker. %Por otro lado $\mathbb{F}^{(p, k)}_\textsf{e} (x)\equiv \mathbb{F}^{(p, k)}_\textsf{e} (x; \vec{\alpha})$ es la fracción del total de eventos que contienen muones de ruido no procedentes del decaimiento de la señal \MSSM\textbf{D}, dada por:
%\begin{equation}\label{fnInverso}
%\mathbb{F}^{(p, k)}_\textsf{e} (x)= 1 - f^{(p, k)}_\textsf{e}  (x; \vec{\alpha})
%\end{equation}
% ~~~ y ~~~ ff^{(p, k)}_\textsf{e} (\vec{\alpha}; x) = 1- f^{(p, k)}_\textsf{e} (\vec{\alpha}; x)
Finalmente, $f^{(p, k)}_\textsf{n} (x; \vec{\alpha}) \equiv f^{(p, k)}_\textsf{n} (x)$ es la fracción de partículas tipo $p$ que se encuentran en eventos con $x$ de estás partículas:
\begin{eqnarray}\label{fn}
f^{(p, k)}_\textsf{n} (x) & = \dfrac{\sum\limits_{i=1}^{N_e} n_i^{(p,k)} \delta_{x,n_i^{(p,k)}}}{\sum\limits_{i=1}^{N_e} \sum\limits_{n=0}^\infty n_i^{(p,k)} \delta_{n, n_i^{(p,k)}}}  = \dfrac{\sum\limits_{i=1}^{N_e} n_i^{(p,k)} \delta_{x, n_i^{(p,k)}}}{\sum\limits_{i=1}^{N_e} n_i^{(p,k)}}
\end{eqnarray}

%\begin{equation}
%\Delta f^{(\geqslant 4\mu, \backsim)}_\textsf{rel} = \left[ \sum_{i,n} \mathbb{E}_i^{(n\mu, \backsim)} - \sum_i \mathbb{E}_i^{(4\mu, \backsim)} \right]/\sum_{i,n} \mathbb{E}_i^{(n\mu, \backsim)}
%\end{equation}
%\begin{equation}
%\Delta f^{(\geqslant 4\mu, \backsim)}_\textsf{n} = \left[ \sum_{i,n} n \cdot \mathbb{E}_i^{(n\mu, \backsim)}- \sum_i 4 \cdot \mathbb{E}_i^{(4\mu, \backsim)}\right]/\sum_{i,n} n \cdot \mathbb{E}_i^{(4\mu, \backsim)}
%\end{equation}
%\begin{equation}
%\Delta f^{(\geqslant 4\mu, \textsf{True})}_\textsf{n} = \left[ \sum_{i} n^{(\mu,\textsf{True})}_i \cdot \mathbb{E}_i^{(\textsf{True})} \cdot (1- \textsf{IO}(n^{(\mu,\textsf{True})}_i,~ 4) \right]/\sum_{i} n^{(\mu,\textsf{True})}_i \cdot \mathbb{E}_i^{(\textsf{True})} 
%\end{equation}

Algunos ejemplos del contenido muónico de los eventos se muestran en la Fig. \ref{contenido_muonico}, donde se pueden visualizar los cambios con la masa del fotón oscuro $m_{\gamma_D}$ y la masa del neutralino oscuro $m_{n_D}$. La caracterización solo se realiza para $m_{n_1}=10$ GeV. Del conjunto de muestras simuladas con \MC ~ sin la reconstrucción del detector ($k=\textsf{True}$), se constató la invarianza de la distribución del contenido muónico $f^{(\mu, \textsf{True})}_\textsf{n} (x; \vec{\alpha}) $ ante los cambios de los parámetros de generación $\vec{\alpha}$, cuestión esperada por la teoría, ya que los muones de procesos de ruido son elementos que no se esperan estar relacionados con el proceso \MSSM\textbf{D} ~ determinado por el decaimiento de la Fig. \ref{fig:sketch_darksector}b.

\begin{figure}[!ht]
\centering
\includegraphics[width=1\textwidth]{Simulacion/imagenes/True_Entries.png}
\caption{(a) Variación del contenido muónico de los eventos antes de pasar por el detector; (b) Variación del porciento de la fracción de muones de ruido con los parámetros de generación $m_{\gamma_D}$ y $M_{n_1}$.}
\label{contenido_muonico}
\end{figure}

De la Fig. \ref{contenido_muonico}a se conoce que el contenido mínimo de muones por evento para $k=\textsf{True}$ es de 4 muones, estos son el resultado de la recreación de la señal \MC ~ proveniente de \MSSM\textbf{D} relacionada con el decaimiento de la Fig. \ref{fig:sketch_darksector}b. Los valores de $f^{(\mu, \textsf{True})}_\textsf{n} (x; \vec{\alpha}) $ con sus respectivos errores se pueden ver en la Tabla \ref{generacion0}, además se concluye de la caracterización de las Fig. \ref{contenido_muonico}b, que la fracción de los muones provenientes de señales de ruido es teóricamente un valor constante, y si valor con un 95\% de confianza está dado por:
\begin{equation}
1- \frac{4 N_e}{\sum n_i^{(p,k)}} = 3.23 \pm 0.19
\end{equation}


% \multicolumn{4}{cccccc}{Parámetros de generación ($\vec{\alpha}$)}
\begin{table}[!h]
\scriptsize
\centering
\begin{tabular}{|c|ccccc|}
\toprule
Variable & $x = 4$ & $x = 5$ & $x = 6$ & $x = 7$ & $x = 8$\\
\midrule
$f^{(\mu, \textsf{True})}_\textsf{n} (x)$ & 
$0.8892 \pm 0.0086$ & $0.0942 \pm  0.0090$ & $0.0161 \pm 0.0016$ & $0.0022 \pm 0.0006$ & $0.0002 \pm 0.0002$ \\
\bottomrule 
\end{tabular}%}
\caption{Fracción de eventos dependiente del contenido muónico. % para un arbitrario parámetro de generación $\vec{\alpha}$.
}
\label{generacion0}
\end{table}

Al analizar los resultados obtenidos, se pudo concluir, que el ruido muónico en la reconstrucción de la señal \MSSM\textbf{D} se encuentra en una fracción del total de eventos dada por $1 - f^{(\mu, \textsf{True})}_\textsf{n} (4; \vec{\alpha}) =  0.113 \pm 0.004 $, fracción no despreciable de nuestro conjunto. Los datos que se poseen no son adecuados para estudiar. Todos los resultados obtenidos  la correspondencia con la masa del neutralino ligero $m_{n_1}$, de aquí que las conclusiones dadas en la sección no la incluyen.

\subsection{Variación de las propiedades de los muones}

Analizar la señal \textbf{Dark}-\SUSY ~o \MSSM\textbf{D}~ mediante las propiedades de los muones sin la reconstrucción del detector dará una base de comparación y un mayor entendimiento de la teoría. Además, separar la información según los muones que provienen del decaimiento $h \rightarrow 2n_1 \rightarrow 2n_D + 2\gamma_D \rightarrow 2n_D + 4\mu$ del resto de los procesos se hace necesario para una mejor interpretación de la reconstrucción conjunta de las señales. Se introduce la notación de las propiedades de una partícula $p= n_1, ~n_D, ~\gamma_D, ~\mu$, siendo la distribución de frecuencia dada por:
\begin{equation}\label{Wpk}
\textsf{W}^{(p,k)} (\chi) \equiv \textsf{W}^{(p,k)} (\chi; \vec{\alpha}) ~~~~~~ \longrightarrow ~~~~~~ \textsf{W}^{(p,k)}_N (\chi) = \dfrac{\textsf{W}^{(p,k)} (\chi)}{ \sum\limits_\chi \textsf{W}^{(p,k)} (\chi)}
\end{equation}
donde $\chi$ hace referencia a la propiedad de interés, estás se pueden ver en la Tabla \ref{propiedades}.

\begin{table}[!h]
\centering
\begin{tabular}{|p{1.2cm}|p{13cm}|}
\toprule
$\chi$ & Definición\\
\midrule
$m$ & Masa invariante\\
$PT$ & Momento de la partícula%, normalmente referenciada por su proyección en dirección transversal \texttt{\textbf{PZ}} $=$ \texttt{\textbf{P}} $\sin \theta$, donde $\theta$ es el ángulo polar, definido como aquel entre el vector momento y la dirección positiva del as, normalmente utilizado ya que no es invariante frente a transformaciones de Lorentz
.\\
$\eta$ & Pseudoapidez, esta representa la coordenada espacial que describe el ángulo de una partícula en relación con el eje del haz. \\%Su ecuación tiene la forma:
%\begin{equation}
%\eta = -\ln \left[ \tan \left( \dfrac{\theta}{2} \right)\right]
%\end{equation}\\[-1cm]
$\phi$ & Ángulo azimutal.\\
$c\tau$ & Tiempo de vida media, esta describe la descomposición de las partículas, se expresa comúnmente en términos de vida media, constante de descomposición o vida media. \\

$D_0$ & Parámetro de impacto transversal, se define como la distancia transversal al eje del haz en el punto de máxima aproximación, donde su signo esta dado de acuerdo al momento angular de la traza alrededor de eje.\\

$D_Z$ & Parámetro de impacto longitudinal, definido como la posición de la coordenada $z$ de la traza en el punto de maximo acercamiento.\\
\begin{small}$\textsf{Sum}Pt$\end{small} & Variable de aislamiento basada en el rastreador de partículas, se define como la suma escalar del $PT$ de las partículas en el plano $\eta \times \phi$ dentro de un cono $\Omega$. Solo existen para $k=\textsf{CMS},  \textsf{HL}$. \\
$Iso$ & Combinación del aislamiento de \textbf{ECAL}, \textbf{HCAL} (ver sección \ref{Experimento_CMS}) y \begin{small}$\textsf{Sum}Pt$\end{small}. Solo existen para $k=\textsf{CMS},  \textsf{HL}$.\\

%\item[-] \texttt{SumPtNeutral}: xxx\\
%\item[-] \texttt{SumPtCharged}: xxx\\
\bottomrule
\end{tabular}
\caption{Propiedades y definiciones de las partículas.}
\label{propiedades}
\end{table}

\begin{figure}[!t]
\centering
\includegraphics[width=.9\textwidth]{Simulacion/imagenes/propiedades_True_notMSSM.png}
\caption{Variación de las distribuciones de los muones de procesos de ruido.}
\label{mu_True}

\includegraphics[width=.9\textwidth]{Simulacion/imagenes/propiedades_True_MSSM.png}
\caption{Variación de las distribuciones de los muones caracteristicos de la señal \MSSM\textbf{D}.}
\label{mu_True2}
\end{figure}

Las distribuciones correspondientes a las propiedades de los muones $\textsf{W}^{(\mu,\textsf{True})}_N (\chi)$ proveniente de procesos alternos al decaimiento  \MSSM\textbf{D} se consideran ruido en esta investigación, sus propiedades se pueden visualizar en la Fig. \ref{mu_True}. Como resultado de su caraterización, se concluyó que la morfología de las distribuciones se mantiene con la variación de los parámetros de generación $\vec{\alpha}$. Además, el dominio para los valores del momento tranversal se extiende hasta $PT= [0, 80]~ \textsf{GeV}$, pero el 98\% de los datos se agrupan para valores $<10~\textsf{GeV}$ como se visualiza en su respectiva distribución.



Las distribuciones de las propiedades de los muones $\textsf{W}^{(\mu,\textsf{True})}_N (PT)$ proveniente del decaimiento  \textbf{Dark-}\SUSY ~ (\MSSM\textbf{D}) se pueden visualizar en la Fig. \ref{mu_True2}. Con la comparación de las distribuciones con la variación de los elementos del parámetro de generación $\vec{\alpha}$, se comprobó la invarianza de la morfología de las distribuciones para los parámetros $\eta$ y $\phi$. Las distribuciones del momento transversal muestran variaciones con el parámetro de generación masa del fotón oscuro $m_{\gamma_D}$ y del neutralino oscuro $m_{n_D}$. Se concluye al comparar con las eficiencias de los detectores $k=\textsf{R2}, \textsf{HL}$, que el aumento de la masa teórica del fotón oscuro permite un aumento de la probabilidad de detección de los muones que decaen de ellos, por el contrario el aumento teórico de la masa del neutralino oscuro dificultará la detección de muones de \MSSM\textbf{D} ya que estos estadisticamente tenderán a menores valores del momento. Los datos que se poseen no son adecuados para estudiar la correspondencia con la masa del neutralino ligero $m_{n_1}$, de aquí que las conclusiones dadas en la sección no la incluyen.


%Las distribuciones muestran que el $\sim 95$\% de los muones correspondientes al decaimiento \MSSM\textbf{D} ~ muestran su dominio para valores $PT < 80 \textsf{ GeV}$, para los muones resultantes de procesos de ruido tenemos $PT < 10 \textsf{ GeV}$. Además, se confirma una relación directa entre los estadísticos medios del momento transversal de los muones con el parámetro de generación masa del fotón oscuro $m_{\gamma_D}$, de forma inversa con el parámetro de masa del neutralino oscuro $m_{n_D}$.


\subsection{Características del fotón oscuro}
La reconstrucción del fotón oscuro $\gamma_D$ predicho por el decaimiento \MSSM\textbf{D} es el motivo principal de estudio de esta investigación. La caracterización de sus propiedades y el cambio de la morfología de los gráficos de frecuencias $\textsf{W}^{(\gamma , \textsf{True})}_N (\chi)$ con el cambio de los parámetros de generación $\vec{\alpha}$, permitirá una comprensión mas completa de los resultados obtenidos con la reconstrucción realizada por los detectores en la configuración Run-2 ($\textsf{R2}$) y Alta Luminosidad ($\textsf{HL}$).


\begin{figure}[!t]
\centering
\includegraphics[width=.9\textwidth]{Simulacion/imagenes/True_PT5.png}
\caption{Variación de las propiedades del fotón oscuro $\gamma_D$ con los parámetros de generación $m_{\gamma_D}$, $m_{n_D}$ y $\tau c_{\gamma_D}$.}
\label{PT_mu_True2}
\end{figure}

Los gráficos de la Fig. \ref{PT_mu_True2} muestra la clara dependencia del momento angular $PT$ y con los parámetros de masa de $\vec{\alpha}$, ya que son la masa del fotón $m_{\gamma_D}$ y su tiempo de vida $c\tau_{\gamma_D}$ son tratados por la teoría como parámetros libres, no hay dependencia directa entre ellas. Hay una correspondencia clara entre los parámetros de impacto $D_\textsf{0}$ y $D_\textsf{Z}$ como se esperaría con el parámetro de generación $c\tau_{\gamma_D}$.


\subsection{Identificando y reconstruyendo el fotón oscuro}

Es de gran interés en esta investigación la creación de una metodología de identificación de di-muones, que pueda discernir entre los muones provenientes de la señal \MSSM\textbf{D}, emparejarlos y reconstruir correctamente el fotón oscuro del cual teóricamente se espera que hayan decaído según el diagrama de la Fig. \ref{fig:sketch_darksector}b. Esta herramienta de identificación, puede crearse, haciendo uso de las redes artificiales neuronales, ya que estas poseen altas capacidades de aprendizaje, generalización, adaptación y tolerancia a fallos, haciéndola una herramienta robusta en el reconocimiento de patrones y objetos.     .

\begin{figure}[!h]
\centering
\includegraphics[width=.9\textwidth]{Simulacion/imagenes/IDENTIFICADOR.png}
\caption{Diagrama de la estructura de la red neuronal dedicada a la identificación de di-muones provenientes del fotón oscuro $\gamma_D$.}
\label{identificador}
\end{figure}

Se crea una red artificial que dadas las propiedades de los di-muones, pueda informar si esta selección proviene o no de un fotón oscuro de \MSSM\textbf{D} (ver Fig. \ref{identificador}). Este problema, es equivalente al perceptrón simple, siendo una de las caracterizaciones más básicas en el área de redes neuronales artificiales.
Para implementar este identificador se hace uso de las paqueterías o herramientas de $\textsf{keras}$ programando en el entorno de \textbf{Python}.

Se hace necesario funciones de activación específicas que incluyan las entradas $x_i$ y las salidas $y_i$, las primeras ante la necesidad de reacondicionamiento ante la gran diferencia de rango de los dominios de las variables $\chi$, las salidas deben ser dadas en forma de probabilidades de tal manera que el sumatoria de las salidas sea normalizada y de esta manera poder imponer criterios de binarización. Dado lo cual, se utilizó la targente hiperbólica para la que conexión entre las capas de entrada con las primeras capas ocultas $x_i \longrightarrow H_1m_1$:
\begin{equation}
f(x)=\dfrac{2}{1+e^{-2x}}-1
\end{equation}
Para las capas de salida $H_km_k \longrightarrow y_i$ se utiliza la función softmax:
\begin{equation}
f(x)_j = \dfrac{e^{Z_j}}{\sum_{k=1}^{K}e^{Z_k}}
\end{equation}
La función de activación utilizada para relacionar todas las capas ocultas es una lineal rectificada, semejante a las mostrada en la ec. \ref{relu}. Para poder caracterizar la precisión del modelo clasificatorio implementado, la relación entre el número de predicciones correctas y el número total de muestras de entrada nos permitirá conocer la eficiencia del clasificador:
\begin{equation}
\textsf{accy} =  \dfrac{\textsf{Número de predicciones correctas}}{\textsf{Numero total de predicciones}}
\end{equation}

Los datos que corresponden a las entrada de la red $x_i$ y las salidas $y_i$ fueron obtenidos de las muestras simuladas con variación en los parámetros de generación $\vec{\alpha}$. La cantidad de capas mostró pocos cambios de mejora en el parámetro de eficiencia $\textsf{accy} $ para $k>2$ (ver Fig. \ref{identificador}), tampoco la cantidad de neuronas por capas, es un factor poco determinante en este caso específico. Se implementa una caracterización para diferentes combinación de parámetros $\chi$ como entradas $x_i$, manteniendo constante la cantidad de épocas, los resultados se muestran en la Tabla 
\ref{ajuste1}.

\begin{table}[!h]
\footnotesize
\centering
\begin{tabular}{|cccccc|c||cccccc|c|}
\toprule
\multicolumn{6}{|c|}{$x_i$ consideradas} &  &
\multicolumn{6}{|c|}{$x_i$ consideradas} &  \\
\midrule
%$PT$ & $T$ & $D0$ & $DZ$ & $Phi$ & $Eta$ & \textsf{accy} &
%$PT$ & $T$ & $D0$ & $DZ$ & $Phi$ & $Eta$ & \textsf{accy} \\
%\midrule
$PT$ & $\phi$ & $\eta$ & $c\tau$ & $D_0$ & $D_Z$  & \textsf{accy} &
$PT$ & $\phi$ & $\eta$ & $c\tau$ & $D_0$ & $D_Z$  & \textsf{accy} \\
\midrule
X &   &   &   &   &   & $0.61 \pm 0.16$ & 
  &   &   & X &   &   & $0.63 \pm 0.05$\\
  & X &   &   &   &   & $0.82 \pm 0.04$ & 
  &   &   &   & X &   & $0.62 \pm 0.07$\\  
  &   & X &   &   &   & $0.90 \pm 0.03$ & 
  &   &   &   &   & X & $0.64 \pm 0.04$\\
\bottomrule
X & X & X &  &  &  & $0.93 \pm 0.01$ &
 & X & X &  &  &  & $0.95 \pm 0.02$\\
\bottomrule 
\end{tabular}%}
\caption{Capacidad del identificador fotónico con variaciones en los parámetros de entrada.}
\label{ajuste1}
\end{table}

\begin{figure}[!h]
\centering
\includegraphics[width=.5\textwidth]{Simulacion/imagenes/acc.png}
\caption{Variación de la precisión del identificador durante el proceso de entrenamiento con las épocas para una configuración de entrada dada por los $x_i = \eta, \phi$.}
\label{identificador0}
\end{figure}

De la interpretación de los resultados de la Tabla \ref{ajuste1} se concluye que las propiedades $PT, ~ c\tau, ~ D_0, ~D_Z$ no son determinantes en la identificación de los di-muones, el origen de estos resultados puede estar dado por la inclusión de casos para tiempos teóricos de vida 0, cuestión no valorada en esta investigación. Por el contrario las propiedades $\eta$ y $\phi$ muestran potencial válidado en el \textsf{accy} > 0.80 razón por la cual son elegidos para formar parte de las entradas del entrenamiento final. 

De concluyó que la creación de una herramienta identificadora de di-muones con las entradas $x_i=(\eta,\phi)$ es la más adecuada encontrada, con un \textsf{accy} $= (0.95 \pm 0.02)$ (ver Fig. \ref{identificador0}) se presenta con bajos errores que la hacen una herramienta suficientemente robusta para una investigación en la que se esperan resultados fiables. La implementación de un entrenamiento de está índole disminuiría el tiempo de cómputo y la fiabilidad de los resultados obtenidos.















