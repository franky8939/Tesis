%Dado lo cual, el objetivo de este proyecto es que se estudie, por medio de simulacion de \MC ~ el modelo teórico \textbf{Dark-}\SUSY, mediante la obtención teórica de las propiedades del fotón oscuro $\gamma_D$ en un entorno  simulado de los detectores que realizarían esta detección en dos  configuraciones ya conocidas del experimento \CMS, en la llamada Run-2 y en la  de Alta Luminosidad. En partícular en esta investigación se cumplen los objetivos partículares:
\begin{itemize}
\item Se crea una herramienta en \textsf{python} para la automatización en paralelo del proceso de simulación y generación de datos correspondiente al decaimiento $h \rightarrow 2n_1 \rightarrow 2n_D + 2\gamma_D \rightarrow 2n_D + 4\mu$ del modelo \textbf{Dark-}\SUSY ~ bajo las configuraciones del detector \CMS ~ correspondiente a Run-2 y Alta Luminosidad.
\item Se crea un método de selección de máximo de eventos $N_e$ basado en regresión polinomial, permitiendo optimizar el proceso de generación para obtener la cantidad de eventos de interés con un mínimo de 4 muones que sea requerida por la investigación. 
\item Se analiza las propiedades de los muones y fotones provenientes de la señal \textbf{Dark-}\SUSY ~ o reconstruida por el detector. Se comparán las distribuciones bajo diferentes condiciones de generación dada por el parámetro $\vec{\alpha}$, mostrando además el aumento del dominio de los valores de la pseudorapidez desde $|\eta|\lesssim 2.4$ a $|\eta|\lesssim 4$ y del momento transversal $P_T<10$ GeV a $P_T<.1$ GeV con la actualización del detector. Además, con la actualización se 
reconstruyen un mínimo de $36\%$ muones más. 
\item Como resultado de la actualización del detector \CMS, se disminuye el error en las distribuciones de masa invariante del fotón oscuro $m_{\gamma_D}$ de $\sim 12\% - 28\%$ y permite reconstruir totalmente hasta un $400\%$ más de ellos, variando este valor con los elementos del parámetro de generación $\vec{\alpha}$.

\item Se implementan 2 métodos generales para la identificación de los di-muones. El primero, basado en el uso de un identificador usando redes neuronales, permitiendo la reconstrucción de un mínimo de $\sim 85\%$ del total de fotones permitido por el detector. El segundo, basado en un proceso de comparación de masas por eventos, permitiendo reconstruir entre $\sim 4\%-45\%$ del total de fotones permitido por el detector. 
\end{itemize}