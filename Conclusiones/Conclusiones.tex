Con el objetivo general de estudiar el decaimiento $h \rightarrow 2n_1 \rightarrow 2n_D + 2\gamma_D \rightarrow 2n_D + 4\mu$ correspondiente al modelo \textbf{Dark-}\SUSY, se creo un grupo de herramientas que facilitan el trabajo de simulación y análisis, entre estas están:
\begin{itemize_f}
\item Un generador programado en \textsf{python} que automatiza la simulación del decaimiento \textbf{Dark-}\SUSY~ utilizando \textsfSmall{ Madgraph5+pythia8+Delphes3} con alta adaptabilidad a los requerimientos del usuario.
\item Una clase basada en \textsfSmall{pyroot} para acceder a la información contenida en los archivos \textsfSmall{*.root}, permitiendo estos ser exportados en un formato \textbf{HDF5}.
\item Un método de selección de máximo de eventos $N_e$ basado en regresión polinomial y en \textbf{RNA}, permitiendo optimizar el proceso de generación para obtener la cantidad de eventos de interés con un mínimo de 4 muones que sea requerida por el estudio 
\item Un identificador de di-muones basado en redes neuronales utilizando la paquetería \textsf{keras} en \textsfSmall{python}.
\end{itemize_f}
Haciendo uso de estas herramientas se generó la base de datos utilizada en la exploración y análisis generales de esta investigación, donde se logra recrear correctamente el decaimiento a dos muones del fotón oscuro predicho por el modelo \textbf{Dark-}\SUSY~ bajo las configuraciones del detector \CMS ~ correspondiente a Run-2 y Alta Luminosidad, con diferentes condiciones en las propiedades dadas por el parámetro $\vec{\alpha}$. Al analizar los resultados, se observa una mejora en la probabilidad de reconstrucción del decaimiento como resultado de la actualización del detector \CMS~%en la obtención de información del decaimiento
fundamentado en aumentos de:
\begin{itemize_f}
\item El dominio de los valores de la pseudorapidez desde $|\eta|\lesssim 2.4$ hasta $|\eta|\lesssim 4$ permitiendo acceder desde un 68\% a un 98\% de total el espectro de la teoría.
\item El espectro del momento transversal desde un $P_T>10$ GeV a $P_T>2$ GeV.
\item La cantidad de eventos con potencial de reconstrucción total del decaimiento entre un $2.1 \lesssim f^{(\mu, \texttt{HL})}_\textsf{e} (4)/f^{(\mu, \texttt{R2})}_\textsf{e} (4) \lesssim 9$ veces.%, permitiendo aumentar las posibilidades de reconstruir completamente el decaimiento.
\item La cantidad de muones provenientes de la teoría y reconstruidos por los detectores entre un $1.3 \lesssim  A^{\mu}_n(\textsfSmall{HL})/A^{\mu}_n(\textsfSmall{R2}) \lesssim 2.8$ veces.
\end{itemize_f}
Se concluye que los valores de masa del neutralino oscuro $m_{n_D}$ son determinantes, junto con la masa del fotón oscuro $m_{\gamma_D}$ y su tiempo de vida $c\tau_{\gamma_D}$, en la reconstrucción de los muones provenientes de la teoría \textbf{Dark-}\SUSY.

Los análisis de los muones reconstruidos por el detector \CMS ~ permitieron analizar la capacidad total de reconstrucción correcta de los fotones oscuros como resultado del emparejamiento de los muones con carga eléctrica diferentes (di-muones) en las condiciones mostradas en la Tabla \ref{fotones_reconstruidos}, resultando en:
\begin{itemize_f}
\item Una reconstrucción entre un $0.024\lesssim  A_\textsf{True}^{\gamma_D} (\textsfSmall{R2})\lesssim 0.287$ del total de fotones predicho por la teoría en la configuración Run-2, mientras que, la configuración Alta Luminosidad reconstruye entre un $0.094\lesssim A_\textsf{True}^{\gamma_D} (\textsfSmall{HL})\lesssim 0.426$.
\item Una disminución en la dispersión de masas obtenidas entre un $\backsim 21\%-28\%$ para el 95\% de las masas obtenidas.

%Las distribuciones de masa invariante de los di-muones emparejados muestran 

\item Las diferencias de las masas invariantes reconstruidas en al menos el 95\% de decaimientos no excede los 2 GeV.  
\end{itemize_f}
Este análisis permitió validar dos métodos de identificación de di-muones sobre el conjunto de datos reconstruidos por el detector. El primero, implementando un identificador basado en redes neuronales $\mathbb{N}_\textsf{RNA}$ y permite la reconstrucción entre $92\% \lesssim \epsilon_{\textsf{RNA}} \lesssim 99\%$ del total de fotones oscuros que son posibles reconstruir con una precisión de $\textsf{acc}_\textsf{RNA}>0.93$. El segundo, se basa en un proceso de comparación de masas por eventos $\mathbb{N}_\textsf{ite}$, permitiendo reconstruir entre $3.5 \% \lesssim \epsilon_{\textsf{ite}}\lesssim 44\%$ de la capacidad total de reconstrucción con una precisión de $\textsf{acc}_\textsf{ite}>0.94$. 

Por lo tanto el método $\mathbb{N}_\textsf{RNA}$ es más eficiente en la reconstrucción de los di-muones que el método iterativo de comparación de masa, pero cada entrenamiento del identificador \textbf{RNA} hará que los resultados de su implementación sean diferentes dada la configuración y datos de entrenamiento a los que la investigación acceda. Por otro lado el método iterativo es mayormente más preciso, es simple en su ejecución y sus resultados pueden ser comparados e implementados de forma estándar en otros estudios semejantes.




%Dado lo cual, el objetivo de este proyecto es que se estudie, por medio de simulacion de \MC ~ el modelo teórico \textbf{Dark-}\SUSY, mediante la obtención teórica de las propiedades del fotón oscuro $\gamma_D$ en un entorno  simulado de los detectores que realizarían esta detección en dos  configuraciones ya conocidas del experimento \CMS, en la llamada Run-2 y en la  de Alta Luminosidad. En partícular en esta investigación se cumplen los objetivos partículares:\item Una clase basada en \textsfSmall{pyroot} para acceder a la información contenida en los archivos \textsfSmall{*.root}, permitiendo estos ser exportados los datos requeridos en un formato \textbf{HDF5}.

