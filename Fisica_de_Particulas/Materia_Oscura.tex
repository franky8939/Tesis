%\begin{textblock}{9}(2.5,-4.5)
%\begin{flushright}
%\setlength{\baselineskip}{15pt}
%\textblockrulecolour{white}
%~
%
%``La discrepancia entre lo que se esperaba y lo que se ha observado ha aumentado a lo largo de los años, y nos estamos esforzando cada vez más por llenar el vacío.''\\[.5cm]
%\textit{Jeremiah P. Ostriker}
%\end{flushright}
%\end{textblock}


Detrás de la materia oscura y la energía oscura, el término oscuro hace referencia al desconocimiento sobre cualquiera de las dos, específicamente del tipo de partículas que las componen. Solo sabemos que no están compuestas de protones, neutrones, electrones o neutrinos. Además, ni la materia oscura ni la energía oscura sienten las fuerzas eléctricas y magnéticas y por tanto no interactúan con la luz, no la emiten ni la absorben. Son inmunes a las ondas electromagnéticas en todas las frecuencias, desde el radio, pasando por la luz visible hasta los rayos gamma, de forma rigurosa el calificativo oscuras no aplica, son transparentes, su existencia es supuesta por porque la gravitación es universal y todo lo que tenga masa-energía crea gravedad.