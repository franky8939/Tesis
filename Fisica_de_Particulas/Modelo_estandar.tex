El \ME~es el formalismo teórico-experimental que, hasta el día de hoy, describe con mayor precisión las partículas elementales y sus interacciones. Los mayores desarrollos que dieron forma al \ME~ se obtuvieron en la segunda mitad del siglo XX con el desarrollo de la Teoría Cuántica de Campos: formulación conjunta de la mecánica cuántica y la mecánica relativista, que es capaz de describir la aniquilación, creación, decaimientos e interacciones de las partículas fundamentales. Los modelos teóricos y observaciones experimentales construyeron una clasificación de las partículas en base a sus propiedades fundamentales como lo son la masa, la carga eléctrica, la carga de color y el espín. Dicha clasificación se muestra en la Fig. \ref{estandar}. 

%En el mundo atómico y subatómico se tratan los problemas con la mecánica cuántica y unido a esto las magnitudes de las energías que se manejan para escudriñar el mundo subnuclear requiere del uso de la mecánica relativista superior en complejidad a la mecánica newtoniana, entonces 


