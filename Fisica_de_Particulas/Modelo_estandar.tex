El modelo estándar (\textbf{ME}) es el formalismo teórico-experimental que hasta el día de hoy describe con mayor precisión las interacciones entre las partículas elementales y los diferentes tipos de fuerzas que experimentan las mismas. Los mayores desarrollos teóricos y descubrimientos experimentales que dieron forma al modelo estándar se tuvieron en la segunda mitad del siglo XX con el desarrollo de la Teoría Cuántica de Campo, fruto del esfuerzo de científicos de todo el mundo, los cuales a partir de los modelos teóricos y observaciones experimentales construyeron una clasificación de las partículas en base a sus propiedades fundamentales como lo son la masa, la carga eléctrica, el espín, entre otras. Dicha clasificación se muestra en la Figura \ref{estandar}. 

En el mundo atómico y subatómico se tratan los problemas con la mecánica cuántica y unido a esto las magnitudes de las energías que se manejan para escudriñar el mundo subnuclear requiere del uso de la mecánica relativista superior en complejidad a la mecánica newtoniana, entonces la formulación conjunta de la mecánica cuántica y la mecánica relativista se expresa adecuadamente en el lenguaje de la Teoría Cuántica del Campo, que es capaz de describir la aniquilación, creación, decaimientos e interacciones de las partículas elementales, asi teorías sobre física de las partículas elementales se describen con el lenguaje de teoría cuántica del campo.

