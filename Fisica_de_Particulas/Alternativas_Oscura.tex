
Para comprender y explicar las irregularidades galácticas se han tratado de realizar medidas directas de la materia oscura para descubrir qué es exactamente pero con escaso éxito, motivo por el cual, investigadores que no creen que exista la materia oscura, han propuesto explicaciones alternativas al comportamiento de las galaxias.

\subsubsection{Modificaciones de la gravedad}

Una explicación alternativa a las cuestiones planteadas por la materia oscura es suponer que las inconsistencias observadas son debidas a una incompleta comprensión de la Gravedad. Para explicar las observaciones, a grandes distancias, las fuerzas gravitacionales son más fuertes de lo que nos indicarían la mecánica newtoniana. 
\begin{itemize}
\item[-] \href{https://en.wikipedia.org/wiki/Modified_Newtonian_dynamics}{\textbf{Teoría de Dinámica newtoniana modificada} (\textbf{MOND}) :}
hipótesis que propone una modificación de las leyes de Newton para tener en cuenta las propiedades observadas de las galaxias, como alternativa a la hipótesis de la existencia de materia oscura en términos de los fenomenos cosmológicos, resolviendo teóricamente esta discrepancia al considerar la fuerza gravitacional experimentada por una estrella en las regiones externas de una galaxia proporcional al cuadrado de su aceleración centrípeta (en oposición a la aceleración centrípeta en sí, como en la segunda ley de Newton), o alternativamente si la fuerza gravitacional variaba inversamente con el radio (en oposición al cuadrado inverso del radio, como en la ley de gravedad de Newton). En \textbf{MOND}, la violación de las leyes de Newton se produce a aceleraciones extremadamente pequeñas, características de las galaxias pero muy por debajo de cualquier cosa que se encuentre típicamente en el Sistema Solar o en la Tierra.

\item[-] \href{https://en.wikipedia.org/wiki/Scalar\%E2\%80\%93tensor\%E2\%80\%93vector_gravity}{\textbf{Tensor-Vector-Escalar de la Gravedad (STVG %, de sus siglas en ingles Scalar-Tensor-Vector Gravity
)}} : es una teoría de la gravedad modificada desarrollada por John Moffat, investigador del Perimeter Institute for Theoretical Physics en Waterloo, Ontario, normalmente referida a la teoría por el acrónimo \href{https://en.wikipedia.org/wiki/Scalar\%E2\%80\%93tensor\%E2\%80\%93vector_gravity}{\textbf{MOG}} (\textbf{MO}dified \textbf{G}ravity). \href{https://en.wikipedia.org/wiki/Scalar\%E2\%80\%93tensor\%E2\%80\%93vector_gravity}{\textbf{STVG}} se ha utilizado con éxito para explicar las curvas de rotación de galaxias, los perfiles de masa de los cúmulos de galaxias, lentes gravitacionales en el cúmulo de bala (Fig. \ref{coalision}a), y observaciones cosmológicas sin la necesidad de materia oscura. En una escala más pequeña, en el Sistema Solar, \href{https://en.wikipedia.org/wiki/Scalar\%E2\%80\%93tensor\%E2\%80\%93vector_gravity}{\textbf{STVG}} no predice ninguna desviación observable de la relatividad general y explica el origen de la inercia.

\item[-] \href{https://ca.wikipedia.org/wiki/Expansi\%C3\%B3_c\%C3\%B2smica_en_escala}{\textbf{Expansión cósmica en escala (SEC) :}} 
se presenta una cosmología física integral, elaborada desde un primer principio explicando efectos de arrastre cósmico que dificulta la disminución de las velocidades en las métricas de la teoria, poseyendo un arrastre cósmico que puede explicar los fenómenos en general atribuido a la materia oscura. Es un acercamiento conformal del tiempo en la relatividad general que requiere la extensión discreta exponencial de la coordenada del tiempo para que sea conforme con la continuidad de la variedad. Esta "teoría" proporciona explicaciones más simples que la del modelo estándar de la cosmología.

%Esto significa que la teoría SEC es un modelo del universo, que asume que las cuatro dimensiones del espacio y del tiempo se amplían en escala. Mientras que las dimensiones espaciales del universo se amplían, el paso del tiempo disminuye imperceptiblemente en pasos minúsculos para preservar una escala constante de distancias medidas, DIST. El cosmos del SEC es eterno sin empezar o el extremo, a este respecto es similar a la Teoría del estado estacionario.

\item[-] \href{https://en.wikipedia.org/wiki/Nonsymmetric_gravitational_theory}{\textbf{La Teoría de gravitación no simétrica (NGT) :}}
presenta una extensión física no trivial de la Relatividad General. La componente antisimétrica del campo fundamental en la NGT, corresponde a un \href{https://en.wikipedia.org/wiki/Kalb\%E2\%80\%93Ramond_field}{campo masivo de Kalb-Ramond} de rango finito, que se identifica con una quinta fuerza gravitacional de tipo repulsivo, y genera una geometría no \href{https://en.wikipedia.org/wiki/Riemann_hypothesis}{Riemanniana}.
La \href{https://en.wikipedia.org/wiki/Nonsymmetric_gravitational_theory}{\textbf{NGT}} concuerda con los datos actuales de aceleración del universo, halos de materia
oscura en las galaxias, lentes gravitacionales, y comportamiento de cúmulos, al igual que con los resultados observacionales estándar, sin necesidad de invocar la dominancia de materia oscura.
\end{itemize}
tales aproximaciones poseen claras dificultades explicando la diferencia en el comportamiento de las distintas galaxias y clústeres, las cuales pueden ser minimizadas considerando cantidades de materia oscura diferentes de la teoría tradicional. Otro problema principal de estas explicaciones alternativas es que no explican las anisotropías del fondo cósmico de microondas que, por otro lado, algunas sí predicen la existencia de materia oscura no bariónica. Otras teorías semejante han sido formaladas y estan en proceso de análisis y validación. Más información en la referencia \cite{rojas_teorigravitacion_2008}.

%En agosto de 2006, un estudio de colisión de cúmulos de galaxias afirmaba demostrar que, incluso en una hipótesis de gravedad modificada, la mayoría de la masa tiene que ser alguna forma de materia oscura demostrando que cuando la materia regular es barrida de un cúmulo, los efectos gravitacionales de la materia oscura (que se pensaba que no interactuaba, aparte de su efecto gravitacional) permanecen. Un estudio afirma que TeVeS puede producir el efecto observado, pero esto continúa necesitando que la mayoría de la masa esté en forma de materia oscura, posiblemente en forma de neutrinos ordinarios. También en la  se afirma que cualitativamente encaja con las observaciones sin necesitar la exótica materia oscura.


\subsubsection{Explicaciones de mecánica cuántica*}

En otra clase de teorías se intenta reconciliar la Gravedad con la Mecánica cuántica y se obtienen correcciones a la interacción gravitacional convencional. En teorías escalar-tensoriales, los campos escalares como el campo de Higgs se acopla a la curvatura dada a través del tensor de Riemann o sus trazas. En muchas de tales teorías, el campo escalar es igual al campo de inflación, que es necesario para explicar la inflación cósmica del Universo después del Big Bang, como el factor dominante de la quintaesencia o energía oscura. Utilizando una visión basada en el Grupo de Renormalización, M. Reuter y H. Weyer han demostrado que la constante de Newton y la constante cosmológica pueden ser funciones escalares en el espacio-tiempo si se asocian las escalas de renormalización a los puntos del espacio-tiempo.

En la teoría de la relatividad de escala Laurent Nottale, el espacio-tiempo es continuo pero no diferenciable, conduciendo a la aparición de una Ecuación de Schrödinger gravitacional. Como resultado, aparecen los efectos de cuantización a gran escala. Esto hace posible predecir correctamente las estructuras a gran escala del Universo sin la necesidad de las hipótesis de la materia oscura.













