Incluso cuando el \ME ~ ha tenido gran éxito en explicar disímiles resultados experimentales,  tiene ciertas cuestiones importantes sin resolver. Entre los problemas encontrados en la teoría estándar está la falta de explicación de los orígenes cuánticos de la gravedad haciendo que la teoría sea por el momento incompatible con la relatividad general. El \ME ~ solo puede explicar el 15.45\% de la material del universo y no considera posible la existencia de masa por parte de los neutrinos (cuestión refutada por los estudios de sus oscilaciones). No explica la presencia excesiva de materia que de antimateria, el modelo predice la creación y aniquilación en cantidades estadísticamente semejantes. Tiene \href{https://en.wikipedia.org/wiki/Hierarchy_problem}{problemas de jerarquía} al introducir partículas con masas a través del proceso de ``ruptura espontánea de simetría electrodébil'' (provocado por el campo de Higgs sobre la simetría de norma $\mathbf{U(1) \otimes SU(2)}$), forzando algunas correcciones cuánticas muy grandes debido a la presencia de partículas virtuales y mucho más grandes que la masa de Higgs real.


%\begin{itemize_f}
%\item[-] \textbf{Gravedad :} no 
%\item[-] \textbf{Materia oscura y energía oscura :} como se pudo constatar anteriormente, solo es posible explicar el 4.9\% de la materia presente en el universo. %Sobre el 95.1% que falta, aproximadamente 
%El 26.8\% de la materia del universo apenas interactúa con los campos del Modelo Estándar. %El resto debería ser energía oscura, una densidad de energía constante para el vacío. 
%Los intentos de explicar la energía oscura en términos de la energía del vacío del Modelo Estándar llevan a un error de 120 órdenes de magnitud.

%\item[-] \textbf{Masa de los neutrinos :} el \ME ~ considera a los neutrinos partículas sin masa, . % Los términos de masa para los neutrinos se pueden añadir a mano al Modelo Estándar, pero esto conduce a nuevos problemas teóricos. (Por ejemplo, los términos de masa deben ser extraordinariamente pequeños).

%\item[-] \textbf{Asimetría de la materia–antimateria :} el \ME ~ predice que la materia y la antimateria deben haber sido creadas en cantidades estadisticamente semejantes, cuestión que si fuera real hubiera aniquilado unas a otras durante el enfriamiento del universo.

%\item[-] \textbf{Problema de jerarquía : } teóricamente se introduce partículas con masas a través de un proceso de ruptura espontánea de simetría electrodébil provocado por el campo de Higgs. Dentro del modelo estándar, la masa de Higgs obtiene algunas correcciones cuánticas muy grandes debido a la presencia de partículas virtuales. Estas correcciones son mucho más grandes que la masa de Higgs real, consideración %. Esto significa que el parámetro de "Masa Desnuda" de Higgs en el modelo estándar debe ser ajustado de tal manera que cancele casi por completo las correcciones cuánticas. Este nivel de ajuste fino está considero como 
%no natural por muchos físicos teóricos.

%\item[-] \textbf{Problema CP fuerte :} teóricamente se puede argumentar que el modelo estándar debe contener un término que rompa la simetría \textbf{CP} (relacionando la materia con la antimateria), pero este no se ha encontrado.
%\item[-] 
%\item[-] 
%\end{itemize_f}