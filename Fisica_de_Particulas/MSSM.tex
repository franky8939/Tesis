
En el año 1973 por Julius Wess y Bruno Zumino presentan un modelo en la física de partículas el cual es conocido con el nombre de Modelo de Wess-Zumino, este es un modelo mínimo supersimétrico con solo un fermión y su super compañero bosón. A pesar de que el modelo de Wess-Zumino no representa un modelo físico real, sirvió para fundamentar ciertos aspectos de los modelos físicos supersimétricos teorizados. 

%Entre las posibles búsquedas para descifrar la composición de la materia oscura, algunas de las más populares son entre las partículas 

El primer modelo supersimétrico compatible con el modelo estándar de la física de partículas llamado \MSSM ~ (\textbf{M}odelo \textbf{M}ínimo \textbf{E}stándar \textbf{S}upersimétrico) este fue enunciado en el año 1981 por Howard Georgi y Savas Dimopoulos, este postulaba la existencia de super compañeros de las partículas del modelo estándar en la región entre $100~GeV$ hasta $1~TeV$, prediciendo su aparición en los experimentos de coalisiones de partículas aceleradas. %Los científicos esperan poder demostrar mediante el \LHC~ la existencia de los super compañeros de las partículas elementales ya conocidas.


El \MSSM ~ no es la única opción posible para la supersimetría más allá del \ME, sino la más económica, en esta teoría se agrega un supercompañero a cada partícula \ME, por lo tanto, introduce el higgsino, thewino, el zino, junto con todos los squarks y sleptons, y nada más. Existen muchas extensiones supersimétricas no mínimas del Modelo Estándar %(que, de hecho, están en mejor forma contra las restricciones experimentales con respecto al \MSSM)
. En principio, se puede construir cualquier \textbf{SSM} (\textbf{S}uper\textbf{S}ymmetry \textbf{M}odel), sin embargo, se deben tener en cuenta varias limitaciones al realizarlo.

La única forma inequívoca de reclamar el descubrimiento de la supersimetría es producir superpartículas en el laboratorio. Debido a que se espera que las superpartículas sean de 100 a 1000 veces más pesadas que el protón, se requiere una gran cantidad de energía para hacer estas partículas que solo se pueden lograr en los aceleradores de partículas. %El Tevatron estaba buscando activamente evidencia de la producción de partículas supersimétricas antes de que se cerrara el 30 de septiembre de 2011. La mayoría de los físicos creen que se debe descubrir la supersimetría en el LHC si es responsable de estabilizar la escala débil. 
%Hay cinco clases de partículas en las que se encuentran los supercompañeros del modelo estándar: squarks, gluinos, charginos, neutralinos y sleptons. Estas superpartículas tienen sus interacciones y desintegraciones posteriores descritas por el \MSSM ~ y cada una tiene firmas características.

El \MSSM ~ impone la \href{https://es.wikipedia.org/wiki/Paridad\_R}{paridad R} para explicar la estabilidad del protón agregando una ruptura de supersimetría al introducir operadores explícitos en el Lagrangiano que se le comunica mediante una dinámica desconocida, significando la presencia de 120 parámetros nuevos en el \MSSM. %La mayoría de estos parámetros conducen a una fenomenología inaceptable, como grandes corrientes neutras que cambian de sabor o grandes momentos dipolares eléctricos para el neutrón y el electrón. Para evitar estos problemas, el MSSM toma toda la ruptura de la supersimetría suave para que sea diagonal en el espacio de sabor y para que todas las nuevas fases de violación de CP desaparezcan.

\subsubsection{Lagrangiano del modelo MSSM.}
Desde el punto de vista experimental, ninguna de las compañeras supersimétricas de las partículas del \ME ~ han sido observadas hasta el momento. Si una teoría es invariante bajo transformaciones supersimétricas, las partículas y sus correspondientes supercompañeras deben tener masas idénticas. %Es decir, si la supersimetría no estuviera rota, deberían existir selectrones con una masa igual a me ~ 0.511 MeV, y lo mismo para los demás sleptones y squarks. Y también deberían existir los gluinos y fotinos sin masa. 
Aunque no se conoce el mecanismo de ruptura de \SUSY, este debe ser implementado de forma de que pueda proveer la solución al problema de jerarquía incluso en presencia del rompimiento de esta. Para ello, las relaciones entre los acoplamientos adimensionales de la teoría antes del rompimiento deben mantenerse. Es por esta razón que el rompimiento de la supersimetría debe ser «suave», y el lagrangiano efectivo
del \MSSM ~ tiene que poder ser escrito como:
\begin{equation}
\mathcal{L} = ~ \mathcal{L}_{SUSY}+\mathcal{L}_{soft}
\end{equation}
donde $\mathcal{L}_{SUSY}$ contiene todas las interacciones de gauge de Yukawa preservando la invariancia supersimétrica. El lagrangiano que rompe \SUSY, $\mathcal{L}_{soft}$, no está completamente determinado y su forma explícita así como el conjunto de parámetros involucrados dependen del mecanismo particular de ruptura de \SUSY ~implementado, de forma general, sin indagar en sus orígenes, se fijan solo pidiendo la
invariancia frente $\mathbf{SU(3)_C} \bigotimes \mathbf{SU(2)_L} \bigotimes \mathbf{U(1)_Y}$ haciendo más fácil mantener la cancelación de las divergencias cuadráticas, los términos términos $soft$ proveen exitosamente las masas de las partículas
supersimétricas, a fin de que sean más pesadas que sus correspondientes compañeras del \ME, y la ruptura espontánea de la simetría electrodébil requerida a bajas energías necesaria para explicar la generación de las masas de las partículas. %Si la escala de masa más grande asociada con los términos $soft$ se llama msoft, las correcciones adicionales no supersimétricas al cuadrado de la masa escalar del Higgs debe anularse en el límite msoft 0.

Debido a que la diferencia de masas entre las partículas conocidas del \ME ~y sus supercompañeras las masas de las partículas supersimétricas no pueden ser demasiado grandes, sino se perdería la solución al problema de jerarquía, pero por otro lado, también existe una razón por la cual las partículas supersimétricas deben ser lo suficientemente pesadas para no haber sido descubiertas hasta ahora. Todas las partículas del \MSSM ~que han sido observadas tienen algo en común: deberían no tener masa en ausencia del rompimiento de la simetría electrodébil. %En particular, las masas de los bosones $W^\pm$, $Z^0$, los quarks y leptones son iguales al producto de constantes de acoplamiento adimensionales por $<H> 174 GeV, mientras que el fotón y el gluón necesitan ser no masivos por la invariancia de gauge electromagnética y de QCD. Por el contrario, todas las partículas del MSSM no descubiertas tienen la propiedad contraria. 
Además cada partícula del \MSSM ~ puede tener un término de masa en el lagrangiano en ausencia del rompimiento de la simetría electrodébil.

En un tratamiento fenomenológico completo todos los parámetros del \MSSM~ deberían dejarse libres y determinarse a partir de los datos observados, y luego de que los parámetros hayan sido medidos, se podría intentar extraer información de la física subyacente que está asociada con escalas de energía mayores a la de los experimentos. Sin embargo, realizar predicciones y análisis fenomenológicos con esta cantidad de parámetros es impracticable, por lo cual es necesario realizar suposiciones para reducir los grados de libertad. Es debido a este motivo que no existe una definición precisa del \MSSM ~y es importante conocer cuales son las suposiciones que se han hecho cuando se realiza un determinado análisis.

\subsubsection{Insuficiencias del modelo MSSM.}
Hay varios problemas con el \MSSM, la mayoría de ellos cayendo en la comprensión de los parámetros que lo componen, algunos de estos son:
\begin{itemize_f}
\item[-] El problema $\mu$: el parámetro de masa de Higgsino $\mu$ aparece como un término en el superpotencial, este debe tener el mismo orden de magnitud que la escala de electroválvula, muchos órdenes de magnitud más pequeños que el de la escala de Planck, que es la escala de corte natural. Los términos de ruptura de la supersimetría suave también deben ser del mismo orden de magnitud que la escala de electrodébil. Esto provoca un problema de naturalidad.
\item[-] La pequeñez de las fases de violación de \textbf{CP}: dado que hasta ahora no se ha descubierto ninguna violación de \textbf{CP} adicional a la predicha por el modelo estándar, los términos adicionales en el lagrangiano del \MSSM ~ deben ser invariantes de \textbf{CP}, por lo que sus fases de violación de \textbf{CP} son pequeñas.
\end{itemize_f}
%\item[-] Universalidad de sabores de masas suaves y términos A: dado que hasta ahora no se ha descubierto una mezcla de sabores adicional a la predicha por el modelo estándar, los coeficientes de los términos adicionales en el \MSSM Lagrangiano deben ser, al menos aproximadamente, invariantes de sabores.
%\item[-] 

\subsubsection{Más alla del modelo MSSM.}
En física de partículas, \textbf{N}\MSSM(\textbf{N}ext-to-\textbf{M}inimal \textbf{S}upersymmetric \textbf{S}tandard \textbf{M}odel.) es una extensión supersimétrica del Modelo Estándar que agrega un supercampo quiral singlete adicional al \MSSM ~ y puede usarse para generar dinámicamente el parámetro $\mu$ resolviendo el problema derivado del mismo.


\subsubsection{Origen de la ruptura de \SUSY.}
Debido a que no se ha observado ninguna de las partículas supersimétricas predichas, si existe \SUSY, debe estar rota, y para mantener la solución al problema de jerarquía incluso en presencia del rompimiento de ésta, el rompimiento debe ser suave, incluyendo términos $soft$ al lagrangiano, para el caso de \textbf{N}\MSSM ~ el rompimiento de SUSY es introducido explícitamente. El rompimiento de una simetría global siempre implica un modo no masivo de Nambu-Goldstone con los mismos números cuánticos que el generador de la simetría rota. %En el caso de la supersimetría global, el generador es la carga fermiónica, por lo tanto la partícula de Nambu-Goldstone tiene que ser un fermión de Weyl no masivo neutro, llamado goldstino. Es claro ahora que el rompimiento espontáneo de la supersimetría requiere la extensión del MSSM. 
El rompimiento espontáneo de SUSY tiene que ocurrir en un ``sector oculto'' de partículas que no tienen acoplamientos directos con los supermultipletes quirales (``sector visible'') del \textbf{N}\MSSM, sin embargo, estos dos sectores comparten algunas interacciones que son las responsables de mediar el rompimiento de la supersimetría desde el sector oculto al visible.









