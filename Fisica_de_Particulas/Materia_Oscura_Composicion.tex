La existencia de la materia oscura es importante en la comprensión del modelo de formación del universo teorizado por el Big Bang y del modo de comportamiento de los cuerpos espaciales, de aquí que los investigadores teoricen sobre su composición. %En la bibliografía científica las diferentes propuestas de materia oscura pueden encontrarse categorizadas por:
%\begin{itemize}
%\item[-] Constitución por bariones: se categorizan en materia oscura bariónica o no bariónica.
%\item[-] Velocidad en lugar de a la temperatura real, lo que indica qué tan lejos se movieron los objetos correspondientes debido a movimientos aleatorios en el universo temprano, antes de que se ralentizaran debido a la expansión cósmica; esta es una distancia importante llamada longitud de transmisión libre (FSL).
%
% su composición por materia barionica ()
%\end{itemize}
%
%, estos han propuesto varias categorías posibles.

\subsubsection{Materia oscura bariónica }
En los primeros a\~nos de estudio del problema de la materia oscura en el Universo, se propuso que esta podría ser materia bariónica y otras partículas ligadas a ellos en forma de objetos compactos considerables pero con una emisión electromagnética muy débil. Los candidatos a materia oscura bariónica se encuentran los gases no luminosos, los objetos compactos y masivos de los halos galácticos (MACHOs) y las enanas marrones.
Sin embargo, múltiples líneas de evidencia contradicen este hecho, ya que contribuyen muy poco a la densidad crítica del
Universo.
%La cantidad total de materia oscura bariónica puede ser calculada a partir de la nucleosíntesis del big bang, y las observaciones del fondo de microondas cósmicas. Ambas indican que la cantidad de esta materia es mucho menor que la cantidad total de materia oscura.
%En el caso de la nucleosíntesis, el problema es que una gran cantidad de materia normal significa un universo joven muy denso, es decir, una conversión eficiente de materia a helio-4 y menos deuterio restante. Si asumimos que toda la materia oscura del universo está formada por bariones, entonces hay muchísimo más deuterio en el universo. Esto puede resolverse si hubiera alguna manera de generar deuterio.
%pero se han realizado grandes esfuerzos desde los años 70 con resultado negativo, generalizando la idea de que no puede crearse dicho elemento.

\subsubsection{Materia oscura no-bariónica} 
En general, la materia oscura se puede clasificar en caliente, tibia o fría. Esta clasificación se hace de acuerdo a la dispersión de velocidades de la partícula en el
momento en que se desacopló del plasma primigenio:
\begin{itemize_f}
\item[-] \textbf{Materia oscura caliente: } partículas no bariónicas que se mueven \href{https://en.wikipedia.org/wiki/Ultrarelativistic_limit}{ultrarrelativistamente}. Estas hacen referencia a una determinada partícula $\chi$ de masa $m_\chi$ con una velocidad relativista al momento de desacoplarse del plasma primigenio, por lo tanto, su temperatura cumple con la condición $T_\chi \gg m_\chi$. 

\item[-] \textbf{Materia oscura fría :} partículas no bariónicas que no se mueven relativistamente al momento de desacoplarse ($v_\chi \sim 0$), por lo cual $T_\chi \ll m_\chi$. 

\item[-] \textbf{Materia oscura templada o tibia :} partículas no bariónicas que se mueven relativistamente, es decir con
características intermedias de las frías y calientes, o sea, con dispersión de velocidades al momento de desacoplarse mayores a las de la materia oscura fría pero menores a las de la materia oscura caliente.
\end{itemize_f}

Algunos de los candidatos a materia oscura más populares en el área de la física de partículas son: 

\begin{itemize}
\item[-] \textbf{\Axiones :} Esta partícula es el bosón pseudo-Goldstone que resulta del rompimiento espontáneo de la simetría Peccei-Quinn. Esta simetría se postula en 1977 en las extensiones del modelo estándar para resolver el problema de la violación carga-paridad (\textbf{CP}) de la interacción fuerte en \QCD. Las observaciones cosmológicas y las mediciones en los acelaradores de partículas acotan la masa del axión a valores de $\lesssim 10^{-2}~eV$ por lo que cae en la categoría de materia oscura fría. Una de las características de los axiones es que dado que tiene interacciones extremadamente débiles con otras partículas, éstas podrían no estar en equilibrio térmico en el Universo temprano. 

Más información en: \url{https://wikimili.com/en/Axion}.

\item[-] \WIMPs ( \textbf{W}eakly \textbf{I}nteracting \textbf{M}assive \textbf{P}articles) : son partículas que se desacoplan siendo no relativistas cuando el Universo tenía una temperatura de $\simeq 1~ GeV$, por lo que caen en la clasificación de materia oscura fría. Las masas
de los \WIMPs ~ abarcan un intervalo de $10 ~ GeV ~ - ~ 1 ~ TeV$. Como su nombre lo indica, es un partícula que interactúa débilmente y 
gravitacionalmente con el resto de las especies del modelo estándar. 

Más información en: \href{https://wikimili.com/en/Weakly_interacting_massive_particles}{\texttt{https://wikimili.com/en/Weakly\_interacting\_massive\_par\-ti\-cles}}

Entre los candidatos se encuentran:
\begin{itemize}
\item 	\LSP (\textbf{L}ightest \textbf{S}upersymmetric \textbf{P}article) : es el nombre genérico dado a la más ligera de las partículas hipotéticas adicionales que se encuentran en los modelos supersimétricos. En modelos con conservación de paridad R, el \LSP ~ es estable; en otras palabras, el \LSP ~ no puede descomponerse en ninguna partícula del \ME~ ya que poseen paridad R opuesta. Agunos ejemplos más conocidos son el sneutrino ligero, el neutralino ligero y el gravitonio.

%Más información en: \href{https://en.wikipedia.org/wiki/Lightest_supersymmetric_particle}{\texttt{https://en.wikipedia.org/wiki/Ligh\-test\_su\-per\-symme\-tric\-\_\-par\-ti\-cle}}.

\item \href{https://en.wikipedia.org/wiki/Kaluza\%E2\%80\%93Klein_theory}{\textbf{LKP} (\textbf{L}ightest {K}aluza-Klein \textbf{P}article) :} son las partículas hipotéticas que cumplen con la teoría de \href{https://en.wikipedia.org/wiki/Kaluza\%E2\%80\%93Klein_theory}{Kaluza-Klein (teoría \textbf{KK})} unificadora de la gravitación y electromagnetismo construida alrededor de la idea de una quinta dimensión más allá de los cuatro habituales del espacio y el tiempo, siendo considerada precursor de la teoría de cuerdas. Algunos de sus candidatos ligeros son el fotón \href{https://en.wikipedia.org/wiki/Kaluza\%E2\%80\%93Klein_theory}{\textbf{KK}} y el neutrino \href{https://en.wikipedia.org/wiki/Kaluza\%E2\%80\%93Klein_theory}{\textbf{KK}}, con masas en la escala $TeV$ (para mas información, ver referencia \cite{servant_is_2003}).

%\item \textbf{LTP} (\textbf{L}ightest \textbf{T}-odd \textbf{P}article) :

\end{itemize}


\item[-] \href{https://es.scribd.com/document/273103231/Dark-Pion-Particles-May-Explain-Universe-s-Invisible-Matter}{\textbf{SIMPs} (\textbf{S}trongly \textbf{I}nteracting \textbf{M}assive \textbf{P}article\textbf{s}) :} se supone que los piones oscuros interactúan mucho más fuertemente entre sí, se sugiere que en el universo primitivo los piones oscuros habrían chocado entre sí, reduciendo la cantidad de materia oscura, pero a medida que el universo se expande, las partículas colisionarían cada vez con menos frecuencia, hasta ahora, cuando se extienden de manera tan delgada que casi nunca se encuentran. En la nueva hipótesis, los piones de materia oscura están formados por quark de materia oscura que se mantienen unidos por gluones de materia oscura. (Los quarks ordinarios están unidos por gluones normales), en esta propuesta el gluón oscuro tendría masa.

%\item[-] 

%\item[-] 

%\item[-] 

%\item[-] 
%
%\item[-] 
%
%\item[-] 
%
%\item[-] 
%
%\item[-] 
%
%\item[-] 
%
%\item[-] 
%
%\item[-] 
%
%\item[-] 
\end{itemize}


 
 
 
 