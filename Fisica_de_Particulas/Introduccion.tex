%\begin{textblock}{9}(2.5,-3)
%\begin{flushright}
%\setlength{\baselineskip}{15pt}
%\textblockrulecolour{white}
%~
%
%``No hay nada que hagan los seres vivos que no pueda entenderse desde el punto de vista de que están hechos de átomos que actúan de acuerdo con las leyes de la física.''\\[.5cm]
%\textit{Richard P. Feynman}
%\end{flushright}
%\end{textblock}

%\framebox[0.8][r]{Bummer, I am too wide}

En este capítulo se introduce a la Física de Altas Energías presentando brevemente algunas características de la materia, las interacciones fundamentales, y como estás son tratadas por el modelo estándar. Se define la materia oscura y se incorporan ejemplos que fundamentan su existencia y posible composición.
Se tratan algunos ejemplos de extensiones del modelo estándar con supersimetría, propuestas, para tratar las dificultades del mismo, entre ellos se hace principal énfasis en los fundamentos del modelo \textbf{Dark-}\SUSY.



% ems: revisar la primera oración, muchos de
%Encontrar los fundamentos del funcionamiento de los objetos materiales que componen la naturaleza ha sido una de las tareas de las que se ha ocupado la humanidad. Esta línea de investigación inicio en la química del siglo XIX con el modelo atómico de Dalton (1803) y pasó a ser parte de la física tras el descubrimiento de la radioactividad por %Antoine Henri 
%Becquerel (1896) y del electrón por Thomson (1906).

%A inicios del siglo XX el área de Física de Partículas Elementales se forma como campo independiente junto con el establecimiento de la composición del núcleo atómico y con el advenimiento de los aceleradores. Ésta se establece entonces como la ciencia que estudia los componentes elementales de la materia y las interacciones entre ellos. También se la conoce como Física de Altas Energías debido a la escala del sistema físico sobre el que se teoriza modelando el comportamiento de la materia.

%De forma general, el área de la cosmológica divide la materia en dos grupos: bariónica y oscura. La Física de Altas Energías describe la materia bariónica según la teoría del Modelo Estándar de Partículas \ME(\textbf{S}tandard \textbf{M}odel),  sin embargo, falla en la descripción de la interacción gravitacional, elemento indispensable para estudiar la materia oscura.