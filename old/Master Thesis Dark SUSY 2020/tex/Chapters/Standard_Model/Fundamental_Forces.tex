There are four fundamental forces at work in the universe: the strong force, the weak force, the electromagnetic force, and the gravitational force. The Standard Model includes the electromagnetic, strong and weak forces and all their carrier particles, and explains well how these forces act on all of the matter particles.

\subsection{Fundamental forces}

The fundamental forces (or fundamental interactions) of physics are the ways that individual particles interact with each other. It turns out that every single interaction observed taking place in the universe can be broken down and described by only four (well, generally four—more on that later) types of interactions:

\begin{itemize}
    \item \textbf{\textcolor{azul50}{Gravity: }} has the farthest reach, but it's the weakest in actual magnitude. It is a purely attractive force which reaches through even the "empty" void of space to draw two masses toward each other. Is described under the theory of general relativity, which defines it as the curvature of spacetime around an object of mass. This curvature, in turn, creates a situation where the path of least energy is toward the other object of mass.
    \item \textbf{\textcolor{azul50}{Electromagnetism :}} is the interaction of particles with an electrical charge. Charged particles at rest interact through electrostatic forces, while in motion they interact through both electrical and magnetic forces, is the most prevalent force in our world, as it can affect things at a reasonable distance and with a fair amount of force.
    \item \textbf{\textcolor{azul50}{Weak Interaction (or Weak Nuclear Force) :}} is a very powerful force that acts on the scale of the atomic nucleus. It causes phenomena such as beta decay. It has been consolidated with electromagnetism as a single interaction called the "electroweak interaction.", is mediated by the W boson (there are two types, the W+ and W- bosons) and also the Z boson.
    \item \textbf{\textcolor{azul50}{Strong Interaction (or Strong Nuclear Force) :}} the strongest of the forces is, keeps the nucleons (protons and neutrons) bound together. In the helium atom, for example, it is strong enough to bind two protons together even though their positive electrical charges cause them to repulse each other, the strong interaction allows particles called gluons to bind together quarks to create the nucleons in the first place. Gluons can also interact with other gluons, which gives the strong interaction a theoretically infinite distance, although it's major manifestations are all at the subatomic level.
\end{itemize}


\subsection{Particle Families}

Fundamental particles are either the building blocks of matter, called fermions, or the mediators of interactions, called bosons. Every elementary particle has associated with it a spin quantum number $s$ (often called the spin number or just the spin), where $s$ is any whole number multiple of a half. Fermions have half integral spin quantum numbers ($1/2$, $3/2$, $5/2$, etc.) and bosons have integral spin quantum numbers ($0$, $1$, $2$, etc.). No spin numbers are possible in between these. 

\begin{figure}
    \centering
    \includegraphics[width=0.7\textwidth]{tex/Chapters/Standard_Model/Imagen/Model.png}
    \caption{Standard model particles.}
    \label{fig:estandard_model}
\end{figure}

Fermions obey a statistical rule described by Enrico Fermi (1901–1954) of Italy, Paul Dirac (1902–1984) of England, and Wolfgang Pauli (1900–1958) of Austria called the exclusion principle. Simply stated, fermions cannot occupy the same place at the same time. (More formally, no two fermions may be described by the same quantum numbers.) Leptons and quarks are fermions, but so are things made from them like protons, neutrons, atoms, molecules, people, and walls. This agrees with our macroscopic observations of matter in everyday life. People cannot walk through walls unless the wall gets out of the way.

Bosons, in contrast, are have no problem occupying the same place at the same time. (More formally, two or more bosons may be described by the same quantum numbers.) The statistical rules that bosons obey were first described by Satyendra Bose (1894–1974) of India and Albert Einstein (1879–1955) of Germany. Gluons, photons, and the W, Z and Higgs are all bosons. As the particles that make up light and other forms of electromagnetic radiation, photons are the bosons we have the most direct experience with. In our everyday experience, we never see beams of light crash into one another. Photons are like phantoms. They pass through one another with no effect.

%All fundamental and composite particles have a spin quantum number s (lowercase). This is associated with a spin angular momentum S (uppercase). The SI unit of angular momentum is the kilogram meter squared per second [kgm2/s] or, equivalently, the joule second [Js], which is much too large for elementary particles. Instead ℏ (h bar), also known as the reduced Planck constant (ℏ = h/2π), is used. For reasons that are beyond the scope of this book, the spin quantum number s (which is just a number) and the spin angular momentum S (which is a number with a unit) are not numerically the same. Instead, they are related by a non-obvious equation.


\begin{table}[h!]
  \begin{center}
    \caption{Standard model particles.}
    \label{tab:table1}
    \begin{tabular}{|c|c|c|c|c|c|c|c||} 
        \hline\hline
        \multicolumn{3}{|c|}{\textbf{Family}} & \textbf{Particle} & \textbf{Spin Number} & \textbf{Charge (e)} & \textbf{Color} & \textbf{Mass}\\
        \hline
        F & Q & $u$ & Up Quarks & $1/2$ & $+2/3$ & $r, g, b$	& $2.16$ \\
        \hline
        E & U & $d$ & Down Quark & $1/2$ & $-1/3$ & $r, g, b$ & $4.67$ \\ 
        \hline
        R & A & $c$ & Charm Quark & $1/2$ & $+2/3$ & $r, g, b$ & $1,27$ \\
        \hline
        M & R & $s$ & Strange Quark & $1/2$ & $-1/3$ & $r, g, b$ & $93$ \\
        \hline
        I & K & $t$ & Top Quark & $1/2$ & $+2/3$ & $r, g, b$ & $172.9$ \\
        \hline
        O &  & $b$ & Bottom Quark & $1/2$ & $-1/3$ & $r, g, b$ & $4.18$ \\
        \hline
        N & L & $e$ & Electron & $1/2$ & $-1$ & none & $\approx 0.51$ \\
        \hline
        S & E & $\mu$ & Muon & $1/2$ & $-1$ & none & $\approx 105.65$ \\
        \hline
          & P &  $\tau$ & Tau & $1/2$ & $-1$ & none & $1776.86$ \\
        \hline
          & T & $v_e$ & Electron Neutrino & $1/2$ & $0$ & none & $<2\times10^{-6}$ \\
        \hline
          & O & $v_\mu$ & Muon Neutrino  & $1/2$ & $0$ & none & $<0.19$ \\
        \hline
          & N & $v_\tau$ & Tau Neutrino & $1/2$ & $0$ & none & $<1.2$ \\
        \hline
          & $-$ & $p$ & Proton & $1/2$ & $+1$ & none & $\approx 938.27$ \\
        \hline
          & $- $& $n$ & Neutron & $1/2$ & $0$ & none & $\approx 939.56$ \\
        \hline
        B & & $g$ & Gluon & $1$ & $0$ & 8 colors & $0$ \\
        \hline
        O & & $\gamma$ & Photon & $1$ & $0$ & none & $0$ \\
        \hline
        S & & $W$ & W Boson & $1$ & $0$ & none & $0$ \\
        \hline
        O & & $Z$ & Z Boson & $1$ & $0$ & none & $0$ \\
        \hline
        N & & $H$ & Higgs Boson & $0$ & $0$ & none & $125.1$ \\
        \hline
        \end{tabular}
  \end{center}
\end{table}

Elementary particles have an intrinsic spin angular momentum $S$. The adjective intrinsic means innate or essential to the thing itself. Elementary particles don't have spin because someone is spinning them. They just spin — or rather, they just have a measurable quantity with the same units as angular momentum. In current physics, elementary particles are featureless — like a mathematical point. In order for something to be perceived as spinning, the thing spinning would need something like a "front" and a "back". Featureless, point particles don't have anything like that. Particle physics is best described with mathematics. Spin is a convenient label for a measurable quality and not a description of reality.


\section{Description of standard model}

The standard model is the name given in the 1970s to a theory of fundamental particles and how thea
\subsection{The elements of the Lagrangian}

The Standard Model of particle physics is a quantum field theory. Therefore, its fundamental elements are quantum fields and the excitations of these fields are identified as particles. For example, the quantised excitation of the electron field is interpreted as an electron. From our viewpoint, it is not only permissible, but even advisable to speak directly of elementary particles instead of field excitations when discussing basic principles of particle physics qualitatively in high school.

The Lagrangian of the standard model is an extremely compact notation. Theoretical particle physicists normally know when to sum over which indices, what different abbreviations and derivatives mean, and when to consider each of the fundamental interactions:
\begin{equation}
    \mathcal{L} = - \dfrac{1}{4}F_{\mu v}F^{\mu v} + i\vec{\psi}{\not D} \psi + \psi_i  Y_{ij} \psi_j \phi + |D_m \phi|^2 - V(\phi)
\end{equation}
n the physics classroom, however, it is very difficult to achieve a deep-level understanding because the required mathematics skills go far beyond high-school level.

$\mathcal{L}$  stands for the Lagrangian density, which is the density of the Lagrangian function L in a differential volume element. In other words, $\mathcal{L}$  is defined such that the Lagrangian L is the integral over space of the density: $L={\int}^{}{{\text{d}}^{3}}x~\mathcal{L}$ . 
In 1788, Joseph–Louis Lagrange introduced Lagrangian mechanics as a reformulation of classical mechanics. It allows the description of the dynamics of a given classical system using only one (scalar) function L=T-V where T is the kinetic energy and V the potential energy of the system. The Lagrangian is used together with the principle of least action to obtain the equations of motion of that system in a very elegant way.

When handling quantum fields, instead of the discrete particles of classical mechanics, the Lagrangian density describes the kinematics and dynamics of the quantum system. Indeed, the Lagrangian density of quantum field theory can be compared to the Lagrangian function of classical mechanics. Hence, it is common to refer to $\mathcal{L}$  simply as 'the Lagrangian'.

The term $-\dfrac{1}{4}F_{\mu v}F^{\mu v}$ is the scalar product of the field strength tensor ${{F}_{\mu \nu}}$  containing the mathematical encoding of all interaction particles except the Higgs boson, where $\mu$ and $v$ are Lorentz indices representing the spacetime components. It contains the necessary formulation for these particles to even exist, and describes how they interact with each other. The contents differ depending on the properties of the interaction particles. For example, photons, the interaction particles of the electromagnetic interaction, cannot interact with each other, because they have no electric charge. Therefore, the contribution of the electromagnetic interaction consists only of a kinetic term, the basis for the existence of free photons. The description of gluons and the weak bosons also includes interaction terms in addition to the kinetic terms. Gluons, for example, are colour-charged themselves and can therefore also interact with each other. This leads to an exciting consequence: the Standard Model of particle physics predicts the existence of bound states consisting only of gluons, so-called 'glueballs'. However, no experiment has detected glueballs thus far.


The term $i\vec{\psi}{\not D} \psi$ describes how interaction particles interact with matter particles. The fields $\psi$ and $\bar{\psi}$  describe (anti)quarks and (anti)leptons. The bar over $\bar{\psi}$  means that the corresponding vector must be transposed and complex-conjugated; a technical trick to ensure that the Lagrangian density remains scalar and real. ${\not D}$  is the so-called covariant derivative, featuring all the interaction particles (except the Higgs), but this time without self-interactions.

The beauty of this term is that it contains the description of the electromagnetic, weak, and strong interactions. Indeed, while all three fundamental interactions are different, the basic vertices by which they can be visualised look quite similar. We will start by discussing the most important interaction of our daily lives, the electromagnetic interaction. Here, pair production or annihilation of electrons and positrons, and the absorption or emission of photons by electrons, are prominent examples. All four of these processes can be represented using Feynman diagrams with the same basic vertex.

This term $\psi_i  Y_{ij} \psi_j \phi$ describes how matter particles couple to the Brout–Englert–Higgs field phgr and thereby obtain mass. The entries of the Yukawa matrix yij represent the coupling parameters to the Brout–Englert–Higgs field, and hence are directly related to the mass of the particle in question. These parameters are not predicted by theory, but have been determined experimentally.

Parts of this term still cause physicists headaches: it is still not clear why neutrinos are so much lighter than other elementary particles, in other words, why they couple only very weakly to the BEH field. In addition, it is still not possible to derive the entries of the Yukawa matrix in a theoretically predictive way.

It is known that particles with high mass, in other words with a strong coupling to the Brout–Englert–Higgs field,also couple strongly to the Higgs boson. This is currently being verified experimentally at the LHC, where Higgs bosons are produced in particle collisions. However, Higgs bosons transform into particle–antiparticle pairs after about 10-22 s. Depending on their mass, i.e. their coupling parameter,certain particle–antiparticle pairs are much more likely, and thus easier to observe experimentally, than others.nThis is because the coupling parameter, which describes the coupling to the Higgs boson, is simply the mass of the particle itself. The Higgs boson is thus more likely to be transformed into pairs of relatively more massive particles and anti-particles. Measurements by the ATLAS detector show, for example, evidence of the direct coupling of the Higgs boson to tauons.

The term $|D_m \phi|^2$ describes how the interaction particles couple to the BEH field. This applies only to the interaction particles of the weak interaction, which thereby obtain their mass. This has been proven experimentally, because couplings of W bosons to Higgs bosons have already been verified. Photons do not obtain mass by the Higgs mechanism, whereas gluons are massless because they do not couple to the Brout–Englert–Higgs field.

The term $V(\phi)$ describes the potential of the BEH field. Contrary to the other quantum fields, this potential does not have a single minimum at zero but has an infinite set of different minima. This makes the Brout–Englert–Higgs field fundamentally different and leads to spontaneous symmetry-breaking (when choosing one of the minima). As discussed for terms 4 and 6, matter particles and interaction particles couple differently to this 'background field' and thus obtain their respective masses. This also describes how Higgs bosons couple to each other. The Higgs boson, the quantised excitation of the BEH field, was experimentally confirmed at CERN in 2012. In 2013, François Englert and Peter Higgs were awarded the Nobel Prize in Physics for the development of the Higgs mechanism.











