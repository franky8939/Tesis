\section{Experiment CMS}
The Compact Muon Solenoid is a general-purpose particle physics experiment, designed to see a wide range of particles and phenomena produced in LHC collisions. These scientists will use the data collected from the complex CMS detector to search for new phenomena including the Higgs boson, supersymmetry, and extra dimensions. They will also measure the properties of previously-discovered quarks and bosons with unprecedented precision, and be on the lookout for completely new, unpredicted phenomena.

\subsection{Compact Muon Solenoid}

The Compact Muon Solenoid (CMS) experiment is one of two large general-purpose particle physics detectors built on the Large Hadron Collider (LHC) at CERN in Switzerland and France. The goal of CMS experiment is to investigate a wide range of physics, including the search for the Higgs boson, extra dimensions, and particles that could make up dark matter.

CMS is 21 metres long, 15 m in diameter, and weighs about 14,000 tonnes, is designed as a general-purpose detector, capable of studying many aspects of proton collisions at 0.9-13 TeV, the center-of-mass energy of the LHC particle accelerator.

The CMS detector is built around a huge solenoid magnet. This takes the form of a cylindrical coil of superconducting cable that generates a magnetic field of 4 teslas, about 100 000 times that of the Earth. The magnetic field is confined by a steel 'yoke' that forms the bulk of the detector's weight of 12 500 tonnes. An unusual feature of the CMS detector is that instead of being built in-situ underground, like the other giant detectors of the LHC experiments, it was constructed on the surface, before being lowered underground in 15 sections and reassembled.

It contains subsystems which are designed to measure the energy and momentum of photons, electrons, muons, and other products of the collisions. The innermost layer is a silicon-based tracker. Surrounding it is a scintillating crystal electromagnetic calorimeter, which is itself surrounded with a sampling calorimeter for hadrons. The tracker and the calorimetry are compact enough to fit inside the CMS Solenoid which generates a powerful magnetic field of 3.8 T. Outside the magnet are the large muon detectors, which are inside the return yoke of the magnet.

\subsection{Description}

The interaction point is the point in the centre of the detector at which proton-proton collisions occur between the two counter-rotating beams of the LHC. At each end of the detector magnets focus the beams into the interaction point. At collision each beam has a radius of 17 μm and the crossing angle between the beams is 285 μrad.

At full design luminosity each of the two LHC beams will contain 2,808 bunches of 1.15×1011 protons. The interval between crossings is 25 ns, although the number of collisions per second is only 31.6 million due to gaps in the beam as injector magnets are activated and deactivated.

At full luminosity each collision will produce an average of 20 proton-proton interactions. The collisions occur at a centre of mass energy of 8 TeV. But, it is worth noting that for studies of physics at the electroweak scale, the scattering events are initiated by a single quark or gluon from each proton, and so the actual energy involved in each collision will be lower as the total centre of mass energy is shared by these quarks and gluons (determined by the parton distribution functions).

The tracker is using of obtein the momentum of particles, this is crucial in helping us to build up a picture of events at the heart of the collision. One method to calculate the momentum of a particle is to track its path through a magnetic field; the more curved the path, the less momentum the particle had. The CMS tracker records the paths taken by charged particles by finding their positions at a number of key points.

The tracker can reconstruct the paths of high-energy muons, electrons and hadrons (particles made up of quarks) as well as see tracks coming from the decay of very short-lived particles such as beauty or “b quarks” that will be used to study the differences between matter and antimatter.

The tracker needs to record particle paths accurately yet be lightweight so as to disturb the particle as little as possible. It does this by taking position measurements so accurate that tracks can be reliably reconstructed using just a few measurement points. Each measurement is accurate to 10 µm, a fraction of the width of a human hair. It is also the inner most layer of the detector and so receives the highest volume of particles: the construction materials were therefore carefully chosen to resist radiation.

The CMS tracker is made entirely of silicon: the pixels, at the very core of the detector and dealing with the highest intensity of particles, and the silicon microstrip detectors that surround it. As particles travel through the tracker the pixels and microstrips produce tiny electric signals that are amplified and detected. The tracker employs sensors covering an area the size of a tennis court, with 75 million separate electronic read-out channels: in the pixel detector there are some 6000 connections per square centimetre.

The CMS silicon tracker consists of 13 layers in the central region and 14 layers in the endcaps. The innermost three layers (up to 11 cm radius) consist of 100×150 μm pixels, 66 million in total.

The next four layers (up to 55 cm radius) consist of 10 cm × 180 μm silicon strips, followed by the remaining six layers of 25 cm × 180 μm strips, out to a radius of 1.1 m. There are 9.6 million strip channels in total.

During full luminosity collisions the occupancy of the pixel layers per event is expected to be $0.1\%$, and $1–2\%$ in the strip layers. The expected HL-LHC upgrade will increase the number of interactions to the point where over-occupancy would significantly reduce trackfinding effectiveness. An upgrade is planned to increase the performance and the radiation tolerance of the tracker.

The Electromagnetic Calorimeter (ECAL) is designed to measure with high accuracy the energies of electrons and photons. This is constructed from crystals of lead tungstate, PbWO4. This is an extremely dense but optically clear material, ideal for stopping high energy particles. Lead tungstate crystal is made primarily of metal and is heavier than stainless steel, but with a touch of oxygen in this crystalline form it is highly transparent and scintillates when electrons and photons pass through it. This means it produces light in proportion to the particle is energy. These high-density crystals produce light in fast, short, well-defined photon bursts that allow for a precise, fast and fairly compact detector. It has a radiation length of χ0 = 0.89 cm, and has a rapid light yield, with $80\%$ of light yield within one crossing time (25 ns). This is balanced however by a relatively low light yield of 30 photons per MeV of incident energy. The crystals used have a front size of 22 mm × 22 mm and a depth of 230 mm. They are set in a matrix of carbon fibre to keep them optically isolated, and backed by silicon avalanche photodiodes for readout.

The ECAL, made up of a barrel section and two "endcaps", forms a layer between the tracker and the HCAL. The cylindrical "barrel" consists of 61,200 crystals formed into 36 "supermodules", each weighing around three tonnes and containing 1700 crystals. The flat ECAL endcaps seal off the barrel at either end and are made up of almost 15,000 further crystals.

For extra spatial precision, the ECAL also contains preshower detectors that sit in front of the endcaps. These allow CMS to distinguish between single high-energy photons (often signs of exciting physics) and the less interesting close pairs of low-energy photons.

At the endcaps the ECAL inner surface is covered by the preshower subdetector, consisting of two layers of lead interleaved with two layers of silicon strip detectors. Its purpose is to aid in pion-photon discrimination.

The Half of the Hadron Calorimeter (HCAL) measures the energy of hadrons, particles made of quarks and gluons (for example protons, neutrons, pions and kaons). Additionally it provides indirect measurement of the presence of non-interacting, uncharged particles such as neutrinos.

The HCAL consists of layers of dense material (brass or steel) interleaved with tiles of plastic scintillators, read out via wavelength-shifting fibres by hybrid photodiodes. This combination was determined to allow the maximum amount of absorbing material inside of the magnet coil.

The CMS magnet is the central device around which the experiment is built, with a 4 Tesla magnetic field that is 100,000 times stronger than the Earth’s. CMS has a large solenoid magnet. This allows the charge/mass ratio of particles to be determined from the curved track that they follow in the magnetic field. It is 13 m long and 6 m in diameter, and its refrigerated superconducting niobium-titanium coils were originally intended to produce a 4 T magnetic field. The operating field was scaled down to 3.8 T instead of the full design strength in order to maximize longevity.

The job of the big magnet is to bend the paths of particles emerging from high-energy collisions in the LHC. The more momentum a particle has the less its path is curved by the magnetic field, so tracing its path gives a measure of momentum. CMS began with the aim of having the strongest magnet possible because a higher strength field bends paths more and, combined with high-precision position measurements in the tracker and muon detectors, this allows accurate measurement of the momentum of even high-energy particles.

The tracker and calorimeter detectors (ECAL and HCAL) fit snugly inside the magnet coil whilst the muon detectors are interleaved with a 12-sided iron structure that surrounds the magnet coils and contains and guides the field. Made up of three layers this “return yoke” reaches out 14 metres in diameter and also acts as a filter, allowing through only muons and weakly interacting particles such as neutrinos. The enormous magnet also provides most of the experiment’s structural support, and must be very strong itself to withstand the forces of its own magnetic field.

The muon detectors, detecting muons is one of CMS’s most important tasks. Muons are charged particles that are just like electrons and positrons, but are 200 times more massive. We expect them to be produced in the decay of a number of potential new particles; for instance, one of the clearest "signatures" of the Higgs Boson is its decay into four muons.

Because muons can penetrate several metres of iron without interacting, unlike most particles they are not stopped by any of CMS's calorimeters. Therefore, chambers to detect muons are placed at the very edge of the experiment where they are the only particles likely to register a signal.

To identify muons and measure their momenta, CMS uses three types of detector: drift tubes (DT), cathode strip chambers (CSC) and resistive plate chambers (RPC). The DTs are used for precise trajectory measurements in the central barrel region, while the CSCs are used in the end caps. The RPCs provide a fast signal when a muon passes through the muon detector, and are installed in both the barrel and the end caps.
