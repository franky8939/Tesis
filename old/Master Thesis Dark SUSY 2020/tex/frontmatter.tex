% !TEX root = ../thesis-sample.tex

% --------- FRONT MATTER PAGES ---------------------
% Title of the thesis
\title{Simulation studies for XXXXXXXX}

% Author name
\author{Francisco}

% Previous degrees
\bsdepartment{XXXXX}
\bsschool{XXXXX}
\bsgrad{May 2009}

%\msdepartment{Aeronautical and Astronautical Engineering}
%\msschool{Purdue University}
%\msgrad{May 2013}
%\showmsdegree % you can show or hide the MS degree line 
\hidemsdegree

% PhD degree commands
%Committee
\showcommitteepage % hide this page if you're doing a MS thesis
%\hidecommitteepage 
\committee{ %
Alfredo M. Castaneda Hernandez, Physics Professor (DIFUS) Universidad de Sonora\\ 
Dissertation Director\\ % remember to add a space between committee members

XXXXXX, Physics Professor (DIFUS) Universidad de Sonora, \\
Committee Member
}

% Chair must be entered separately for formatting reasons.
\chair{A. Castaneda}
\chairtitle{Physics Professor}
% Department
\department{Department of Physics Research}

%\phdgrad{XXXX, 2020}
%\defensedate{December 1, 2018}
% Year of completion for copyright page and perhaps other places
%\year=2020

% Copyright page
%\copyrightholder{Someone else}

% Dedication
\dedication{ %
Include a fancy quote or dedication
}

% Acknowledgments
\acknowledgments{
    Here you can acknowledge all of those people who have helped you to reach this point.
}

% -----------------------------------------------------------------
% Typically only one of Preface/Foreward/Prologue would be in your thesis.
% To choose one simply delete the others and they will automatically dissappear

% Preface

%\preface{
%    This is the preface. 
%    It's another front matter page that offers additional detail into your work.
%    Typically, only one (preface OR prologue OR foreword) is used. 
%    You can remove the other sections by deleting them inside %\texttt{tex/frontmatter.tex} or using the appropriate show or hide commands.
%}

%\prologue{
%    This is the prologe. 
%    It's another front matter page that offers additional detail into your work.
%    Typically, only one (preface OR prologue OR foreword) is used. 
%    You can remove the other sections by deleting them inside %\texttt{tex/frontmatter.tex} or using the appropriate show or hide commands.
%}

%\foreword[2]{
%    This is the forword. 
%    It's another front matter page that offers additional detail into your work.
%    Typically, only one (preface OR prologue OR foreword) is used. 
%    You can remove the other sections by deleting them inside %\texttt{tex/frontmatter.tex} or using the appropriate show or hide commands.
%}
% ----------------------------------------------------------------------

% commands to show or hide front matter pages

%\showcopyright
\showabstract
\showcommitteepage
\showdedication
\showacknowledgments
%\showpreface
%\showprologue
%\showforeword

% ------------ TABLE OF CONTENTS ----------------------
% Commands to hide or show lists of figures, tables, etc.
\showlistoffigures
\showlistoftables
\hidenomenclature

% --------- ACRONYMS and SYMBOLS ------------------------------
% TODO Deprecate the entire acronym package and switch to glossaries

% You can either use the acronymn or glossaries package (both work)
% Definition of any abbreviations used.
\abbreviations{
    \acro{CRTBP}{Circular Restricted Three Body Problem}
    \acro{NSA}{National Security Agency}
    \acro{SSME}{Space Shuttle Main Engine}
}
% call an abbreviation using \ac{abbrev}

% symbols and acronyms only show up when used in the text
\symbols{
    \acro{J}{Moment of Inertia}
}       

% if you want acronymn (simpler) then change these to show
\hidelistofabbreviations
\hidelistofsymbols

% if you want glossaries (more powerful) then leave above as hide
% GLOSSARIES package options - automatically turns off front pages from acronym package

% acronymns and symbols are basically the same, but there are two provided 
% locations where they can show up
\setabbreviationstyle[acronym]{long-short}
\setabbreviationstyle[abbreviation]{long-short}
\makeglossaries
% you can hide/show the glossaries page
\showglossarieslistofabbreviations
\showglossarieslistofsymbols
\showglossariesglossaryofterms



% acronyms defined in glossaries
\newabbreviation{crtbp}{CRTBP}{Circular Restricted Three Body Problem}
\newabbreviation{lidar}{LIDAR}{Light Detection and Ranging}
% defining abbreviations like this allows for autocompletion
\newglossaryentry{filo}{
    name={FILO},
    type=\glsxtrabbrvtype,
    description={first in last out},
    first={first in last out (FILO)}
}

% glossary entries
\newglossaryentry{linux}{
    name=Linux,
    description={is a generic term referring to the family of Unix-like computer operating systems that use the Linux kernel},
    plural=Linuces
}

\newglossaryentry{matrix}{
    name={matrix},
    plural={matrices},
    description={rectangular array of quanttities}
}

% symbols
\newglossaryentry{M}{
    type=symbols,
    name={\ensuremath{M}},
    sort=M,
    description={a \gls{matrix}}
}

\newglossaryentry{F}{
    type=symbols,
    name={\ensuremath{F}},
    sort=F,
    description={External Force}
}
% Some abstract text
\abstract{
This is the abstract. 


\lipsum[1]
}

% La concentración esta en el orden rojo, verde y azul
\definecolor{azul50}{rgb}{0.2,0.30,0.50} 
\definecolor{gris50}{rgb}{0.02,0.05,0.10} 
\definecolor{gris10}{rgb}{0.80,0.80,0.80} 
\definecolor{verde}{rgb}{0,0.50,0}
\definecolor{morado}{rgb}{0.6,.3,.5}



