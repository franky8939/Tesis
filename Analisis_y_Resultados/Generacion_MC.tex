La simulación de los distintos procesos físicos y la respuesta del detector a los mismos es necesaria para poder optimizar y estimar el desempeño de los diferentes análisis. Además, permite que las estrategias utilizadas en la identificación de partículas puedan ser desarrolladas con anterioridad a la toma de datos y las eficiencias de los algoritmos pueden ser puestos a prueba. La preparación de las búsquedas de nueva física necesitan una simulación detallada del detector para estimar su potencial de descubrimiento y para desarrollar métodos óptimos para medir las propiedades de las partículas.

Es fundamental un correcto entendimiento de los procesos de señal y de fondo para poder distinguir entre ambos. Una vez que los datos de colisiones reales están disponibles, los datos simulados también resultan necesarios para poder encontrar desviaciones del \ME.
La estructura de los eventos de colisiones de altas energías son realmente complejos y no predecibles de primeros principios. Los generadores de eventos permiten separar el problema en varios pasos más simples, algunos de los cuales pueden ser descriptos por primeros principios, y otros necesitan ser basados en modelos apropiados con parámetros ajustados a los datos. Un aspecto central de los generadores es que proveen una descripción del estado final para poder construir cualquier observable y compararlos con los datos de colisiones reales.
