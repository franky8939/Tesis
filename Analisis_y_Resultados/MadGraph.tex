Las colisiones de alta energía entre partículas elementales normalmente dan lugar a estados finales complejos, con grandes multiplicidades de hadrones, leptones, fotones y neutrinos. La relación entre estos estados finales y la descripción física subyacente no es simple, por dos razones principales:
\begin{itemize_f}
\item[-] No se posee una comprensión completa de la física a implementar.
\item[-] Cualquier enfoque analítico se vuelve intratable por las grandes multiplicidades.
\end{itemize_f}

La forma de abordar este problemática es generando eventos completos por los métodos de \MC, la complejidad se domina mediante una subdivisión del problema completo en un conjunto de tareas separadas más simples, simulando todos los aspectos principales de los eventos: selección de procesos duros, la radiación de estado inicial y final, los restos de haces, la fragmentación, las desintegraciones, el cálculo de secciones transversales y su coincidencia con generadores de eventos, etc. Esto resulta en eventos que deben ser directamente comparables con los observables experimentalmentes y de esta forma programas pueden usarse para extraer la física de las comparaciones con los datos existentes, o para estudiar la física en experimentos futuristas.

Con el objetivo de refundir un análisis \LHC ~siendo una de sus herramientas mas importantes desarrollada por el proyecto y la solución a los problemas anteriormente planteado tenemos a MadGraph5\_aMC@NLO (ver referencia \cite{alwall_automated_2014}) siendo un framework que tiene como objetivo proporcionar todos los elementos necesarios para la fenomenología del \ME ~ y extensiones, permitiendo el uso de una variedad de herramientas relevantes para generación, manipulación y análisis de eventos. 

%Toda la información con respecto a su instalación y configuración se puede encontrar en su página oficial: \url{https://twiki.cern.ch/twiki/bin/view/CMSPublic/MadgraphTutorial}. 

La salida del mismo son archivos $*.lhe$ o \textbf{LHEF} (\textbf{L}es \textbf{H}ouches \textbf{E}vent \textbf{F}ile), estos datos son los que obtenemos de un generador \MC (\textbf{M}onte \textbf{C}arlos) como MadGraph. Esta salida contiene varios parámetros cinemáticos de todas las partículas involucradas en los procesos junto con la descripción de procesos simulados, parámetros de modelo y condiciones de ejecución. El análisis con \textbf{LHEF} se realiza para comprender varias propiedades cinemáticas básicas de la muestra de \MC ~ producida. Las variables cinemáticas asociadas con diferentes partículas del evento se pueden obtener utilizando este método.


El principal conjunto de herramientas que componen la herramienta Mad\-Graph5\_\-a\-MC\-@NLO, o a las que puede ser integrada son: Delphes (\cite{de_favereau_delphes_2014-1}), MadAnalysis4 y MadAnalysis5 (\cite{conte_madanalysis_2013}), ExRootAnalysis, Golem95 (\cite{binoth_precise_2008}), QCDLoop (\cite{ellis_scalar_2008}), maddm (\cite{wang_novel_2018}), maddump (\cite{buonocore_event_2019}), pythia8 (\cite{sjostrand_introduction_2015-1}), lhapdf5 y lhapdf6 (\cite{buckley_lhapdf6_2015}), collier (\cite{denner_collier_2017}), hepmc, mg5amc\_py8\_interface (\cite{sjostrand_introduction_2015-1}), ninja (\cite{hirschi_tensor_2016, peraro_ninja_2014, mastrolia_integrand_2012}), oneloop (\cite{van_hameren_oneloop_2011}). Su implementación se hace necesaria para estudios de partículas, dada su versatilidad, aunque sea una herramienta de altas exigencias en conocimiento de programación y trabajo en el sistema Linux. 

Para uso futuro como parte de esta investigación se profundizará en las herramientas Pythia8 y Delphes, estás a pesar de poderse ejecutar de forma independiente pueden ser integradas con facilidad dentro del programa de Madgraph y de esta manera planificar la receta de nuestro proceso a reconstruir.
